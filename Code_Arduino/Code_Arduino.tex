%	Auteur: Surdez Quentin
%	Titre:	 Étudiant MCT
%	Date: Mai 2022
%	Sujet: Rapport de projet P2213	
%============================================================
\documentclass[
	a4paper,									% paper format
	11pt,										% fontsize
	twoside,									% double-sided
	openright,									% begin new chapter on right side
	notitlepage,									% use no standard title page
	parskip=half,								% set paragraph skip to half of a line
]{scrreprt}										% KOMA-script report
%---------------------------------------------------------------------------

\raggedbottom
\KOMAoptions{cleardoublepage=plain}						%Add header and footer on blank pages

%Load Standard Packages: 
%---------------------------------------------------------------------------
\usepackage[standard-baselineskips]{cmbright}				%Fonts for math equation
\usepackage[french]{babel}							%French hyphenation
	%\usepackage[latin1]{inputenc}  					% Unix/Linux - load extended character set (ISO 8859-1)
	%\usepackage[applemac]{inputenc}
	%\usepackage[ansinew]{inputenc}  					% Windows - load extended character set (ISO 8859-1)
%\usepackage[utf8]{inputenc}							%Translate input in latex language
\usepackage[T1]{fontenc}							%Allow a good hyphenation with accentuated language
%\usepackage{tgbonum}
\usepackage{ae}									%Allow vectorial letters
\usepackage{lmodern}
\usepackage{fancyhdr}								%Allow manipulation on headers and tops
\usepackage{graphicx}								%Integration of images
\usepackage{float}									%Better integration of floating objects(tables, etc)
\usepackage{caption}								%For captions of figures and tables
\usepackage{booktabs}								%Nicer tables
\usepackage{tocvsec2}								%Means of controlling the sectional numbering
\usepackage{verbatim}								%Integration of source code
\usepackage{moreverb}								%Extension of verbatim
\usepackage{listings}								%Integration of code in LATEX 
\usepackage{multirow}								%Tables with multiple rows
\usepackage{pdfpages}						
\usepackage{pst-all}								%Better handling of texts and images
\usepackage{mathrsfs}
\usepackage{colortbl}
\usepackage{listings}
\usepackage{minted}
\captionsetup{font=small}



\newcolumntype{R}[1]{>{\raggedleft\arraybackslash }b{#1}}
\newcolumntype{L}[1]{>{\raggedright\arraybackslash }b{#1}}
\newcolumntype{C}[1]{>{\centering\arraybackslash }b{#1}}		%Adding new column types

%---------------------------------------------------------------------------

%Load Math packages
%---------------------------------------------------------------------------
\usepackage{amsmath}                    				   	% various features to facilitate writing math formulas
\usepackage{amsthm}                       	 				% enhanced version of latex's newtheorem
\usepackage{amsfonts}                      					% set of miscellaneous TeX fonts that augment the standard CM
										
\usepackage{amssymb}							% mathematical special characters
\usepackage{exscale}							% mathematical size corresponds to textsize
\usepackage{listings}
\usepackage{tikz,pgfplots}
\usepackage{array}
%---------------------------------------------------------------------------

%QR Code
%---------------------------------------------------------------------------
\usepackage{qrcode}
\usepackage{subcaption}							%Add sub-caption easily
%---------------------------------------------------------------------------

% Package to facilitate placement of boxes at absolute positions
%---------------------------------------------------------------------------
%\usepackage[absolute]{textpos}
\usepackage[absolute,overlay]{textpos}
\setlength{\TPHorizModule}{1mm}
\setlength{\TPVertModule}{1mm}
%---------------------------------------------------------------------------	

% Definition of Colors
%---------------------------------------------------------------------------
\RequirePackage{color}							% Color (not xcolor!)
\definecolor{linkblue}{rgb}{0,0,0.8}            				% Standard
\definecolor{darkblue}{rgb}{0,0.08,0.45} 				% Dark blue
\definecolor{brickred}{cmyk}{0,0.89,0.94,0.28} 			% Brickred
\definecolor{linkcolor}{rgb}{0,0,0}        					% Black for the print-version!
\definecolor{PEjaune}{rgb}{1,0.84,0}        				% Jaune PE
\definecolor{PEvert}{rgb}{0.14,0.5,0}        				% Vert PE

\definecolor{VertVAUD}{rgb}{0.054, 0.662, 0.301} %14, 169, 77}

%---------------------------------------------------------------------------

% Hyperref Package (Create links in a pdf)
%---------------------------------------------------------------------------
\usepackage[
	pdftex,frenchb,bookmarks,plainpages=false,pdfpagelabels,
	backref = {false},							% No index backreference
	colorlinks = {true},							% Color links in a PDF
	hypertexnames = {true},						% no failures "same page(i)"
	bookmarksopen = {true},						% opens the bar on the left side
	bookmarksopenlevel = {0},					% depth of opened bookmarks
	pdftitle = {Rapport Projet P2213},		   		% PDF-property
	pdfauthor = {Surdez Quentin},        				% PDF-property
	pdfsubject = {Promotion 21-22},        				% PDF-property
	linkcolor = {linkcolor},              					% Color of Links
	citecolor = {linkcolor},              					% Color of Cite-Links
	urlcolor = {linkcolor},               					% Color of URLs
]{hyperref}
%---------------------------------------------------------------------------

% Set up page dimension
%---------------------------------------------------------------------------
\usepackage[
	a4paper,
	left=28mm,
	right=15mm,
	top=30mm,
	headheight=20mm,
	headsep=10mm,
	textheight=242mm,
	footskip=15mm
]{geometry}
\setlength\parindent{20pt}
%---------------------------------------------------------------------------

% Makeindex Package
%---------------------------------------------------------------------------
\usepackage{makeidx}                         					% To produce index
\makeindex                                    					% Index-Initialisation
%---------------------------------------------------------------------------

% Intro:
\pgfplotsset{compat=1.18} 
%---------------------------------------------------------------------------
\begin{document}                              					% Start Document
\settocdepth{subsection}									% Set depth of toc
\pagenumbering{Roman}														
%---------------------------------------------------------------------------

%Set up header and footer
%---------------------------------------------------------------------------
\fancyhf{}												%clean all fields
\fancypagestyle{plain}{									%new definition of plain style
	\fancyfoot[OR, EL]{\footnotesize \thepage}			%footer right part --> page number
	\fancyfoot[OL, ER]{\footnotesize \leftmark}			%footer left part --> chapter
	\fancyfoot[CE, CO]{P2213, QS \& RD}
	\fancyhead[C]{
	\begin{textblock}{0}[0, 0](5, 8)						%header center part --> logo CPNV + MCT 
		\includegraphics[scale=0.7]{img/logoCPNV.png}
	\end{textblock}
	\begin{textblock}{0}[0, 0](175, 3)
		\includegraphics[scale=0.5]{img/logoMCT.jpg}
	\end{textblock}
	}
}

\renewcommand{\chaptermark}[1]{\markboth{\thechapter.  #1}{}}
\renewcommand{\headrulewidth}{0pt}				% no header stripline
\renewcommand{\footrulewidth}{0pt} 				% no bottom stripline
\renewcommand\listoflistingscaption{Liste des codes sources}


\pagestyle{plain}
\let\cleardoublepage\clearpage
%---------------------------------------------------------------------------

%=============================================================================================
% Page principale
%=============================================================================================
%---------------------------------------------------------------------------
\begin{titlepage}
	\setlength{\unitlength}{1mm}
%	\begin{textblock}{230}(-10,-10)
%		\begin{picture}(230,35)%32)
%			\put(73,0){\color{VertVAUD}\rule{160mm}{40mm}}
%		\end{picture}
%	\end{textblock}

	\begin{textblock}{0}[0,0](5,12) % (x,y)
		\includegraphics[scale=1]{img/logoCPNV.png}
	\end{textblock}

	\begin{textblock}{0}(158, 2)
		\includegraphics[scale=0.8]{img/logoMCT.jpg}
	\end{textblock}





% Titre / Sous-titre / Auteur / Image de garde:
%---------------------------------------------------------------------------
	
	\flushleft
	\vspace*{1cm}
	%\fontfamily{cmr}\selectfont			%To have the default font
	\fontsize{18pt}{20pt}\selectfont
	CPNV - Centre Professionnel du Nord Vaudois \\
	\fontsize{12pt}{15pt}\selectfont\vspace{0.5em}
	MCT - Modules complémentaires techniqeus

	\vspace{3cm}

	\fontsize{30pt}{32pt}\selectfont 
	\noindent \textbf{Explication code Arduino} \\

	\fontsize{18pt}{20pt}\selectfont\vspace{0.3em} P2213 \\

	\vspace{4cm}
	\fontsize{12pt}{15pt}\selectfont
	\begin{tabbing}
		xxxxxxxxxxxxxxx\=xxxxxxxxxxxxxxxxxxxxxxx \kill
		Rédacteur:\> Quentin Surdez\\ \\
		Relecture:\> Rafael Dousse\\ \\
		École:\> CPNV\\ \\
		Date:\> Yverdon-Les-Bains, le \today \\
	\end{tabbing}
\end{titlepage}
%---------------------------------------------------------------------------

%===========================================
% Table des matières
%===========================================
\tableofcontents

\listoffigures									% Table des figures
%\listoftables									% Table des tableaux
\listoflistings % Now typeset the list
\cleardoublepage
%---------------------------------------------------------------------------

%=============================================================================================
% Introduction
%=============================================================================================
\pagenumbering{arabic}
\setcounter{page}{1}

\chapter{Introduction}
Ce document a pour but d'expliciter les commentaires et l'architecture du code de l'Arduino Nano pour le contrôle des moteurs. Il permet un approfondissement du code et de comprendre la logique derrière 
les différents choix effectués.

%=============================================================================================
% Document
%=============================================================================================

\chapter{Asservissement des moteurs}

L'asservissement de moteurs à courant continu est au coeur de notre projet. En effet, chaque fonction que nous construisons par la suite reposera totalement ou en partie sur la fonction d'asservissement. \par

L'asservissement permet de contrôler un certain paramètre dans un système. Il se caractérise par le besoin d'un système de maintenir une consigne donnée, qu'importe les perturbations infligées au système.
Nous avons donc asservi notre système au niveau des moteurs. En effet, nous souhaitons que le robot aille tout droit lorsqu'on le lui demande, ce qui se traduit par le besoin d'avoir
la même vitesse constante aux deux moteurs. \par

\section{PID}



Pour asservir nos moteurs à courant continu, nous avons choisi d'utiliser la méthode des facteurs PID. L'acronyme PID signifie proportionnel, intégral, dérivateur. Ce sont les 3 facteurs que nous 
utilisons pour construire notre asservissement. \par

Pour utiliser les différents facteurs, nous devons premièrement calculer l'erreur. Elle est la différence entre la consigne et la vitesse observée. Nous appliquons le facteur proportionnel directement à l'erreur.
Le facteur intégratif est appliqué à la somme des erreurs sur le temps. Enfin, le facteur dérivateur s'applique sur la différence des erreurs sur le temps. \par

	\begin{center}

L'équation pour le calcul de la valeur de contrôle est la suivante : 
		\begin{figure}[h]
			\[u=k_pe + k_i \int e \cdot dt + k_d \cfrac{de}{dt}\] 
			\caption{Équation du PID}
			\label{eq1}
		\end{figure}
	\end{center} 

Les différents calculs se font par rapport à un $dt$, cette particularité est très importante et a permis de diriger le développement du code. Pour mieux comprendre l'intégration de l'équation dans le code 
une image permet de visualiser le processus : 

	\begin{figure}[h!]
		\centering
		\includegraphics[scale=.3]{img/PID.png}
		\label{PIDfig}
		\caption{Intégration de l'erreur et du PID}
	\end{figure}



\chapter{Interruptions}

Comme expliqué au-dessus, la composante du temps est très importante. C'est grâce à elle que l'asservissement peut être fonctionnel et robuste. Pour intégrer ce besoin, il existe une fonctionnalité 
des microcontrôleurs MegaAVR appelée interruption. Les interruptions permettent, comme leur nom l'indique, d'interrompre le programme pour effectuer une tâche bien précise. Souvent, cette tâche est une incrémentation de valeur, ainsi 
l'interruption ne dure qu'un très court instant. Après avoir fini la routine d'interruption, le programme reprend exactement là où il en était. \par

\section{Les interruptions attachées}

La librairie Arduino permet d'utiliser une fonction appelée attachInterrupt(). Cette fonction permet de donner à un pin la responsabilité de lancer une routine d'interruption. Cette interruption sera alors attachée à un pin. Ainsi, si un pin est activé 
à un internal de temps régulier, la routine d'interruption se fera à un interval de temps régulier. \par

Cela a été mis en place pour notre projet en incorporant un Arduino gérant exclusivement le temps. Toutes les $x$ secondes, un signal est envoyé à l'Arduino se chargeant de calculer le PID. Un schéma permet de visualier 
le processus : 
\begin{figure}
	\centering
	\includegraphics[scale=0.2]{img/interruptions_attachees.jpg}
	\label{IntSch}
	\caption{Schéma des interruptions attachées}
\end{figure}

\chapter{Code}
pygme
Le code a été écrit dans le but de répondre au besoin de notre projet. Les fonctions le composant et la logique appliquée
seront discutés dans les prochains sous-chapitres. Premièrement, une explication du code nécessaire au setup de la
communication I2C, puis donner de plus grandes explications que les commentaires sur les fonctions créées. Enfin, une
une explication sur la remise à zéro des différentes valeurs pour que les différentes fonctions puissent intéragir entre
elles sans compromettre l'intégrité de l'ensemble.

\section{Communication I2C}


\begin{listing}[!ht]
\usemintedstyle{rainbow_dash}
\begin{minted}{c}
// Set up du programme tourner sur soi ---------------------------
void tourneSurSoi(){

  
  attachInterrupt(digitalPinToInterrupt(photoElectricSensor), compteurDistance, RISING);
  attachInterrupt(digitalPinToInterrupt(pin_PID), asservissement, RISING);  

 
  consigne_moteur = 2;
  consigne_moteur1 = 0;

  tourner = 1;

}
\end{minted}
\caption{Fonction du robot en C}
\label{listing:2}
\end{listing}


\section{Fonctions appelées}



\section{Remise à zéro des valeurs}






\end{document}