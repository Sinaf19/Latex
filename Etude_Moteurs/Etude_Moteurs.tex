%	Auteur: Surdez Quentin
%	Titre:	 Étudiant MCT
%	Date: Mai 2022
%	Sujet: Rapport de projet P2213	
%============================================================
\documentclass[
	a4paper,									% paper format
	11pt,										% fontsize
	twoside,									% double-sided
	openright,									% begin new chapter on right side
	notitlepage,									% use no standard title page
	parskip=half,								% set paragraph skip to half of a line
]{scrreprt}										% KOMA-script report
%---------------------------------------------------------------------------

\raggedbottom
\KOMAoptions{cleardoublepage=plain}						%Add header and footer on blank pages

%Load Standard Packages: 
%---------------------------------------------------------------------------
\usepackage[standard-baselineskips]{cmbright}				%Fonts for math equation
\usepackage[french]{babel}							%French hyphenation
	%\usepackage[latin1]{inputenc}  					% Unix/Linux - load extended character set (ISO 8859-1)
	%\usepackage[applemac]{inputenc}
	%\usepackage[ansinew]{inputenc}  					% Windows - load extended character set (ISO 8859-1)
%\usepackage[utf8]{inputenc}							%Translate input in latex language
\usepackage[T1]{fontenc}							%Allow a good hyphenation with accentuated language
%\usepackage{tgbonum}
\usepackage{ae}									%Allow vectorial letters
\usepackage{lmodern}
\usepackage{fancyhdr}								%Allow manipulation on headers and tops
\usepackage{graphicx}								%Integration of images
\usepackage{float}									%Better integration of floating objects(tables, etc)
\usepackage{caption}								%For captions of figures and tables
\usepackage{booktabs}								%Nicer tables
\usepackage{tocvsec2}								%Means of controlling the sectional numbering
\usepackage{verbatim}								%Integration of source code
\usepackage{moreverb}								%Extension of verbatim
\usepackage{listings}								%Integration of code in LATEX 
\usepackage{multirow}								%Tables with multiple rows
\usepackage{pdfpages}						
\usepackage{pst-all}								%Better handling of texts and images
\usepackage{mathrsfs}
\usepackage{colortbl}


\newcolumntype{R}[1]{>{\raggedleft\arraybackslash }b{#1}}
\newcolumntype{L}[1]{>{\raggedright\arraybackslash }b{#1}}
\newcolumntype{C}[1]{>{\centering\arraybackslash }b{#1}}		%Adding new column types

%---------------------------------------------------------------------------

%Load Math packages
%---------------------------------------------------------------------------
\usepackage{amsmath}                    				   	% various features to facilitate writing math formulas
\usepackage{amsthm}                       	 				% enhanced version of latex's newtheorem
\usepackage{amsfonts}                      					% set of miscellaneous TeX fonts that augment the standard CM
										
\usepackage{amssymb}							% mathematical special characters
\usepackage{exscale}							% mathematical size corresponds to textsize
\usepackage{listings}
\usepackage{tikz,pgfplots}
\usepackage{array}
%---------------------------------------------------------------------------

%QR Code
%---------------------------------------------------------------------------
\usepackage{qrcode}
\usepackage{subcaption}							%Add sub-caption easily
%---------------------------------------------------------------------------

% Package to facilitate placement of boxes at absolute positions
%---------------------------------------------------------------------------
%\usepackage[absolute]{textpos}
\usepackage[absolute,overlay]{textpos}
\setlength{\TPHorizModule}{1mm}
\setlength{\TPVertModule}{1mm}
%---------------------------------------------------------------------------	

% Definition of Colors
%---------------------------------------------------------------------------
\RequirePackage{color}							% Color (not xcolor!)
\definecolor{linkblue}{rgb}{0,0,0.8}            				% Standard
\definecolor{darkblue}{rgb}{0,0.08,0.45} 				% Dark blue
\definecolor{brickred}{cmyk}{0,0.89,0.94,0.28} 			% Brickred
\definecolor{linkcolor}{rgb}{0,0,0}        					% Black for the print-version!
\definecolor{PEjaune}{rgb}{1,0.84,0}        				% Jaune PE
\definecolor{PEvert}{rgb}{0.14,0.5,0}        				% Vert PE

\definecolor{VertVAUD}{rgb}{0.054, 0.662, 0.301} %14, 169, 77}

%---------------------------------------------------------------------------

% Hyperref Package (Create links in a pdf)
%---------------------------------------------------------------------------
\usepackage[
	pdftex,frenchb,bookmarks,plainpages=false,pdfpagelabels,
	backref = {false},							% No index backreference
	colorlinks = {true},							% Color links in a PDF
	hypertexnames = {true},						% no failures "same page(i)"
	bookmarksopen = {true},						% opens the bar on the left side
	bookmarksopenlevel = {0},					% depth of opened bookmarks
	pdftitle = {Rapport Projet P2213},		   		% PDF-property
	pdfauthor = {Surdez Quentin},        				% PDF-property
	pdfsubject = {Promotion 21-22},        				% PDF-property
	linkcolor = {linkcolor},              					% Color of Links
	citecolor = {linkcolor},              					% Color of Cite-Links
	urlcolor = {linkcolor},               					% Color of URLs
]{hyperref}
%---------------------------------------------------------------------------

% Set up page dimension
%---------------------------------------------------------------------------
\usepackage[
	a4paper,
	left=28mm,
	right=15mm,
	top=30mm,
	headheight=20mm,
	headsep=10mm,
	textheight=242mm,
	footskip=15mm
]{geometry}
%---------------------------------------------------------------------------

% Makeindex Package
%---------------------------------------------------------------------------
\usepackage{makeidx}                         					% To produce index
\makeindex                                    					% Index-Initialisation
%---------------------------------------------------------------------------

% Intro:
\pgfplotsset{compat=1.18} 
%---------------------------------------------------------------------------
\begin{document}                              					% Start Document
\settocdepth{subsection}									% Set depth of toc
\pagenumbering{roman}														
%---------------------------------------------------------------------------

%Set up header and footer
%---------------------------------------------------------------------------
\fancyhf{}												%clean all fields
\fancypagestyle{plain}{									%new definition of plain style
	\fancyfoot[OR, EL]{\footnotesize \thepage}			%footer right part --> page number
	\fancyfoot[OL, ER]{\footnotesize \leftmark}			%footer left part --> chapter
	\fancyfoot[CE, CO]{P2213, QS \& RD}
	\fancyhead[C]{
	\begin{textblock}{0}[0, 0](5, 8)						%header center part --> logo CPNV + MCT 
		\includegraphics[scale=0.7]{img/logoCPNV.png}
	\end{textblock}
	\begin{textblock}{0}[0, 0](175, 3)
		\includegraphics[scale=0.5]{img/logoMCT.jpg}
	\end{textblock}
	}
}

\renewcommand{\chaptermark}[1]{\markboth{\thechapter.  #1}{}}
\renewcommand{\headrulewidth}{0pt}				% no header stripline
\renewcommand{\footrulewidth}{0pt} 				% no bottom stripline

\pagestyle{plain}
\let\cleardoublepage\clearpage
%---------------------------------------------------------------------------

%=============================================================================================
% Page principale
%=============================================================================================
%---------------------------------------------------------------------------
\begin{titlepage}
	\setlength{\unitlength}{1mm}
%	\begin{textblock}{230}(-10,-10)
%		\begin{picture}(230,35)%32)
%			\put(73,0){\color{VertVAUD}\rule{160mm}{40mm}}
%		\end{picture}
%	\end{textblock}

	\begin{textblock}{0}[0,0](5,12) % (x,y)
		\includegraphics[scale=1]{img/logoCPNV.png}
	\end{textblock}

	\begin{textblock}{0}(158, 2)
		\includegraphics[scale=0.8]{img/logoMCT.jpg}
	\end{textblock}





% Titre / Sous-titre / Auteur / Image de garde:
%---------------------------------------------------------------------------
	
	\flushleft
	\vspace*{1cm}
	%\fontfamily{cmr}\selectfont			%To have the default font
	\fontsize{18pt}{20pt}\selectfont
	CPNV - Centre Professionnel du Nord Vaudois \\
	\fontsize{12pt}{15pt}\selectfont\vspace{0.5em}
	MCT - Modules complémentaires techniqeus

	\vspace{3cm}

	\fontsize{30pt}{32pt}\selectfont 
	\noindent \textbf{Étude choix des moteurs} \\

	\fontsize{18pt}{20pt}\selectfont\vspace{0.3em} P2213 \\

	\vspace{4cm}
	\fontsize{12pt}{15pt}\selectfont
	\begin{tabbing}
		xxxxxxxxxxxxxxx\=xxxxxxxxxxxxxxxxxxxxxxx \kill
		Rédacteur:\> Quentin Surdez\\ \\
		Relecture:\> Rafael Dousse\\ \\
		École:\> CPNV\\ \\
		Date:\> Yverdon-Les-Bains, le 10 mars 2022\\%\today \\
	\end{tabbing}
\end{titlepage}
%---------------------------------------------------------------------------

%===========================================
% Table des matières
%===========================================
\tableofcontents

%\listoffigures									% Table des figures
%\listoftables									% Table des tableaux
\cleardoublepage
%---------------------------------------------------------------------------

%=============================================================================================
% Introduction
%=============================================================================================
\pagenumbering{arabic}
\setcounter{page}{1}

\chapter{Introduction}
Cette étude permet de mettre en avant les caractéristiques des moteurs à notre disposition. Un tableau récapitulatif permet de voir un résumé de ces caractéristiques. \\ 

\chapter{Moteurs}
\section{Moteur DC L149}
\includegraphics[scale=.7]{img/Moteur_L149.png}
\vspace{1cm}

Avantages : \begin{itemize}
			\item Moteur accessible dans la salle de cour avec une extension lui permettant d’être vissé sur les plaques de test
			\item Facilité à utiliser pour des prototypes
			\item Pèse 55gr
			\item Tire jusqu'à 85mA avec le torque maximum
		\end{itemize}

\vspace{1cm}

Désavantages : \begin{itemize}
			\item Couple maximum de 2.5 Ncm
			\item RPM à 80 avec le couple maximum et 105 à moteur libre
		\end{itemize}

\section{Moteur DC ModelCraft}
\includegraphics*[scale=.5]{img/Moteur_MODELCRAFT.png}
\vspace{1cm}

Avantages : \begin{itemize}
			\item Couple nominal de 9.8 Ncm
			\item Couple de décrochage de 215.6 Ncm
			\item Moteur accessible dans la salle de cours avec une extension lui permettant d'être vissé sur les plaques de test
			\item RPM à 317 avec le couple maximum et 333 à moteur libre
		\end{itemize}

\vspace{1cm}

Désavantages : \begin{itemize}
			\item Courant nominal de 1.6A
			\item Plus bruyant que les autres
		\end{itemize}

\newpage
\section{Moteur pas à pas}
\includegraphics*[scale=.3]{img/Moteur_PAS.png}
\vspace{1cm}

Avantages : \begin{itemize}
			\item Courant de 1.2A
			\item Permet de garder la position grâce à un couple de maintien de 39.2 Ncm
   			\item Couple de détente de 1960 Ncm
		\end{itemize}

\vspace{1cm}

Désavantages : \begin{itemize}
			\item 360gr
			\item Nécessite un double pont H pour un moteur, donc 2 doubles pont H pour contrôler nos deux roues
   			\item Plus encombrant que les autres moteurs 
		\end{itemize}
		
\section{Moteur AC}
\includegraphics*[scale=.7]{img/Moteur_AC.jpg}
\vspace{1cm}

Avantages : \begin{itemize}
			\item Uniquement des champs magnétiques et peu de mécanique, donc pas d'usure ni de frottements
		\end{itemize}

\vspace{1cm}

Désavantages : \begin{itemize}
			\item Nécessite un ondulateur pour transformer le courant continu de la batterie et le transformer en courant alternatif
			\item Couple de 0.38 Ncm
   			\item Moteur à commander et non disponible dans la salle de cours 
		\end{itemize}

\newpage
\chapter{Tableau récapitulatif}
\includegraphics[scale=.8]{img/Tableaurecap.png}
\chapter{Conclusion}
 Nous avons choisi les moteurs DC ModelCraft pour leur couple de découplage qui nous permettra d'avoir un robot plus lourd 
que le moteur L149. Nous l'avons aussi choisi, car un moteur DC nécessite un pont H ce qui encombre peu notre robot. \\
 Nous devrons faire attention à bien modéliser notre batterie au vu du courant nominal du moteur choisi.






\end{document}                              					% End 
%=====================

















