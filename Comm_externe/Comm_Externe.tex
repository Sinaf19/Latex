%	Auteur: Surdez Quentin
%	Titre:	 Étudiant MCT
%	Date: Mai 2022
%	Sujet: Rapport de projet P2213	
%============================================================
\documentclass[
	a4paper,									% paper format
	11pt,										% fontsize
	twoside,									% double-sided
	openright,									% begin new chapter on right side
	notitlepage,									% use no standard title page
	parskip=half,								% set paragraph skip to half of a line
]{scrreprt}										% KOMA-script report
%---------------------------------------------------------------------------

\raggedbottom
\KOMAoptions{cleardoublepage=plain}						%Add header and footer on blank pages

%Load Standard Packages: 
%---------------------------------------------------------------------------
\usepackage[standard-baselineskips]{cmbright}				%Fonts for math equation
\usepackage[french]{babel}							%French hyphenation
	%\usepackage[latin1]{inputenc}  					% Unix/Linux - load extended character set (ISO 8859-1)
	%\usepackage[applemac]{inputenc}
	%\usepackage[ansinew]{inputenc}  					% Windows - load extended character set (ISO 8859-1)
%\usepackage[utf8]{inputenc}							%Translate input in latex language
\usepackage[T1]{fontenc}							%Allow a good hyphenation with accentuated language
%\usepackage{tgbonum}
\usepackage{ae}									%Allow vectorial letters
\usepackage{lmodern}
\usepackage{fancyhdr}								%Allow manipulation on headers and tops
\usepackage{graphicx}								%Integration of images
\usepackage{float}									%Better integration of floating objects(tables, etc)
\usepackage{caption}								%For captions of figures and tables
\usepackage{booktabs}								%Nicer tables
\usepackage{tocvsec2}								%Means of controlling the sectional numbering
\usepackage{verbatim}								%Integration of source code
\usepackage{moreverb}								%Extension of verbatim
\usepackage{listings}								%Integration of code in LATEX 
\usepackage{multirow}								%Tables with multiple rows
\usepackage{pdfpages}						
\usepackage{pst-all}								%Better handling of texts and images
\usepackage{mathrsfs}
\usepackage{colortbl}


\newcolumntype{R}[1]{>{\raggedleft\arraybackslash }b{#1}}
\newcolumntype{L}[1]{>{\raggedright\arraybackslash }b{#1}}
\newcolumntype{C}[1]{>{\centering\arraybackslash }b{#1}}		%Adding new column types

%---------------------------------------------------------------------------

%Load Math packages
%---------------------------------------------------------------------------
\usepackage{amsmath}                    				   	% various features to facilitate writing math formulas
\usepackage{amsthm}                       	 				% enhanced version of latex's newtheorem
\usepackage{amsfonts}                      					% set of miscellaneous TeX fonts that augment the standard CM
										
\usepackage{amssymb}							% mathematical special characters
\usepackage{exscale}							% mathematical size corresponds to textsize
\usepackage{listings}
\usepackage{tikz,pgfplots}
\usepackage{array}
%---------------------------------------------------------------------------

%QR Code
%---------------------------------------------------------------------------
\usepackage{qrcode}
\usepackage{subcaption}							%Add sub-caption easily
%---------------------------------------------------------------------------

% Package to facilitate placement of boxes at absolute positions
%---------------------------------------------------------------------------
%\usepackage[absolute]{textpos}
\usepackage[absolute,overlay]{textpos}
\setlength{\TPHorizModule}{1mm}
\setlength{\TPVertModule}{1mm}
%---------------------------------------------------------------------------	

% Definition of Colors
%---------------------------------------------------------------------------
\RequirePackage{color}							% Color (not xcolor!)
\definecolor{linkblue}{rgb}{0,0,0.8}            				% Standard
\definecolor{darkblue}{rgb}{0,0.08,0.45} 				% Dark blue
\definecolor{brickred}{cmyk}{0,0.89,0.94,0.28} 			% Brickred
\definecolor{linkcolor}{rgb}{0,0,0}        					% Black for the print-version!
\definecolor{PEjaune}{rgb}{1,0.84,0}        				% Jaune PE
\definecolor{PEvert}{rgb}{0.14,0.5,0}        				% Vert PE

\definecolor{VertVAUD}{rgb}{0.054, 0.662, 0.301} %14, 169, 77}

%---------------------------------------------------------------------------

% Hyperref Package (Create links in a pdf)
%---------------------------------------------------------------------------
\usepackage[
	pdftex,frenchb,bookmarks,plainpages=false,pdfpagelabels,
	backref = {false},							% No index backreference
	colorlinks = {true},							% Color links in a PDF
	hypertexnames = {true},						% no failures "same page(i)"
	bookmarksopen = {true},						% opens the bar on the left side
	bookmarksopenlevel = {0},					% depth of opened bookmarks
	pdftitle = {Rapport Projet P2213},		   		% PDF-property
	pdfauthor = {Surdez Quentin},        				% PDF-property
	pdfsubject = {Promotion 21-22},        				% PDF-property
	linkcolor = {linkcolor},              					% Color of Links
	citecolor = {linkcolor},              					% Color of Cite-Links
	urlcolor = {linkblue},               					% Color of URLs
]{hyperref}
%---------------------------------------------------------------------------

% Set up page dimension
%---------------------------------------------------------------------------
\usepackage[
	a4paper,
	left=30mm,
	right=30mm,
	top=30mm,
	headheight=20mm,
	headsep=10mm,
	textheight=242mm,
	footskip=15mm
]{geometry}
\setlength\parindent{20pt}
%---------------------------------------------------------------------------

% Makeindex Package
%---------------------------------------------------------------------------
\usepackage{makeidx}                         					% To produce index
\makeindex                                    					% Index-Initialisation
%---------------------------------------------------------------------------

% Intro:
\pgfplotsset{compat=1.18} 
%---------------------------------------------------------------------------
\begin{document}                              					% Start Document
\settocdepth{subsection}									% Set depth of toc
\pagenumbering{roman}														
%---------------------------------------------------------------------------

%Set up header and footer
%---------------------------------------------------------------------------
\fancyhf{}												%clean all fields
\fancypagestyle{plain}{									%new definition of plain style
	\fancyfoot[OR, EL]{\footnotesize \thepage}			%footer right part --> page number
	\fancyfoot[OL, ER]{\footnotesize \leftmark}			%footer left part --> chapter
	\fancyfoot[CE, CO]{P2213, QS \& RD}
	\fancyhead[C]{
	\begin{textblock}{0}[0, 0](10, 8)						%header center part --> logo CPNV + MCT 
		\includegraphics[scale=0.7]{img/logoCPNV.png}
	\end{textblock}
	\begin{textblock}{0}[0, 0](175, 3)
		\includegraphics[scale=0.5]{img/logoMCT.jpg}
	\end{textblock}
	}
}

\renewcommand{\chaptermark}[1]{\markboth{\thechapter.  #1}{}}
\renewcommand{\headrulewidth}{0pt}				% no header stripline
\renewcommand{\footrulewidth}{0pt} 				% no bottom stripline

\pagestyle{plain}
\let\cleardoublepage\clearpage
%---------------------------------------------------------------------------

%=============================================================================================
% Page principale
%=============================================================================================
%---------------------------------------------------------------------------
\begin{titlepage}
	\setlength{\unitlength}{1mm}
%	\begin{textblock}{230}(-10,-10)
%		\begin{picture}(230,35)%32)
%			\put(73,0){\color{VertVAUD}\rule{160mm}{40mm}}
%		\end{picture}
%	\end{textblock}

	\begin{textblock}{0}[0,0](5,12) % (x,y)
		\includegraphics[scale=1]{img/logoCPNV.png}
	\end{textblock}

	\begin{textblock}{0}(158, 2)
		\includegraphics[scale=0.8]{img/logoMCT.jpg}
	\end{textblock}





% Titre / Sous-titre / Auteur / Image de garde:
%---------------------------------------------------------------------------
	
	\flushleft
	\vspace*{1cm}
	%\fontfamily{cmr}\selectfont			%To have the default font
	\fontsize{18pt}{20pt}\selectfont
	CPNV - Centre Professionnel du Nord Vaudois \\
	\fontsize{12pt}{15pt}\selectfont\vspace{0.5em}
	MCT - Modules complémentaires techniqeus

	\vspace{3cm}

	\fontsize{30pt}{32pt}\selectfont 
	\noindent \textbf{Communication externe} \\

	\fontsize{18pt}{20pt}\selectfont\vspace{0.3em} P2213 \\

	\vspace{4cm}
	\fontsize{12pt}{15pt}\selectfont
	\begin{tabbing}
		xxxxxxxxxxxxxxx\=xxxxxxxxxxxxxxxxxxxxxxx \kill
		Rédacteur:\> Quentin Surdez\\ \\
		Relecture:\> Rafael Dousse\\ \\
		École:\> CPNV\\ \\
		Date:\> Yverdon-Les-Bains, le 10 mars 2022\\%\today \\
	\end{tabbing}
\end{titlepage}
%---------------------------------------------------------------------------

%===========================================
% Table des matières
%===========================================
\tableofcontents

%\listoffigures									% Table des figures
%\listoftables									% Table des tableaux
\cleardoublepage
%---------------------------------------------------------------------------

%=============================================================================================
% Introduction
%=============================================================================================
\pagenumbering{arabic}
\setcounter{page}{1}

\chapter{Introduction}
Ce document a pour but d'expliquer notre construction de la communication externe du robot. 
La communication externe se compose du Raspberry PI, de l'Arduino Nano ainsi que d'un 
objet connecté via un réseau WIFI à notre Raspberry PI. Nous discuterons de nos approches pour utiliser la communication
et de comment nous avons résolu les différents problèmes que nous avons rencontrés. 

%=============================================================================================
% Document
%=============================================================================================
\chapter{HotSpot}

Nous avons tout d'abord voulu faire de notre Raspberry PI un hotspot WIFI. Un hotspot est un endroit
qui permet de se connecter, la plupart du temps à internet, via le WIFI. Le WIFI est un protocole de
communication par ondes radio qui permet de relier ensemble plusieurs appareils électroniques. \par

\section{Apache}

\begin{figure}[!ht]
	\centering
	\includegraphics[scale=.2]{img/Apache.png}
	\label{Apache}
	\caption{Logo du serveur Web Apache}
\end{figure}

Apache est la première alternative que nous avons essayé. C'est un serveur Web HTTP qui aurait permis d'héberger notre web
application. Nous l'avons peu à peu pris en main et nous avons réussi à utiliser le Raspberry en hotspot. Cependant, Apache 
restait relativement obscure pour nous. Les différentes possibilités qu'Apache offrait ne correspondait pas à nos besoins. 
Nous ne souhaitions pas héberger sur le web une application, nous ne souhaitions pas faire de notre Raspberry PI un serveur. \par

\section{ROS}

Nous avons eu une autre alternative pour faire de notre Raspberry PI hotspot. Le système d'exploitation ROS, pour Robotic Operating System, 
nous offrait cette possibilité, tout en étant pensé et developpé pour des robots. Nous avons pris connaissance des possibilités 
que cet OS offrait, comme la création de noeud de communications. Cependant, nous avons vite compris que ROS était principalement
fait pour des robots industriels ou de qualité supérieure au notre. Après plusieurs tentatives, nous n'avons pas réussi à installer 
l'OS sur notre Raspberry PI. Nous nous sommes éloignés de l'idée de paramétré notre Raspberry PI en hotspot. \par

\chapter{Réseau}

Nous nous sommes tournés vers une alternative. Le partage de connexion. En effet, plusieurs appareils connectés au même
réseau peuvent communiquer entre eux via adresse IP et port. Les adresses IP sont assignés pour chaque appareil sur 
le réseau. Les ports logiciels sont ceux utilisés principalement par les protocoles de communications standards comme TCP ou 
HTTP. Nous pouvons assigné une web app à une adresse IP spécifique et un port spécifique. Ce qui nous permet d'y accéder via 
un appareil connecté au même réseau que le Raspberry PI. \par

Le nouveau challenge avec cette méthode était d'avoir une page web intéractive dont nous pouvions recevoir les informations
et les transmettre à notre script en Pyhton. Donc de passer à du HTML, Javascript à du Python pour pouvoir envoyer via I2C 
à noter Arduino. \par

\section{Flask}

\begin{figure}[!ht]
	\centering
	\includegraphics[scale=.3]{img/Flask.png}
	\vspace{.5cm}
	\label{Flask}
	\captionabove{Logo du micro-framework Flask}
\end{figure}

Nous avons découvert via un tutoriel \href{https://pyimagesearch.com/2019/09/02/opencv-stream-video-to-web-browser-html-page/}{pyimagesearch.com}
la possibilité d'utiliser le micro-framework Flask pour facilement intéragir avec une web application via notre script Python.
Un framework est un logiciel facilitant l'intéraction avec un utilisateur, donc facilitant la création d'applications. 
Flask est un micro-framework car il ne nécessite aucune librairies ou fichier additionnel pour fonctionner. Des extensions
existent pour, par exemple, lui ajouter une fonctionnalité de database avec SQLAlchemy. \par

Ainsi, Flask nous a permis de communiquer avec une page web et de communiquer l'information donnée sur cette page 
à notre script Pyhton. L'information récupérée peut être utilisée dans notre programme. 
Ce protocole est celui que nous avons choisi pour notre communication externe. Nous pouvons nous intéresser de plus 
près sur la strucutre de notre web app. \par

\chapter{Apprentissage HTML/CSS}

\section{Site Web}

Le site web a été pensé pour être en premier lieu pratique. Nous avons d'abord un menu qui nous permet de choisir 
quel mode nous souhaitons.


\end{document}