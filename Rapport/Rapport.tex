%	Auteur: Surdez Quentin
%	Titre:	 Étudiant MCT
%	Date: Mai 2022
%	Sujet: Rapport de projet P2213	
%============================================================
\documentclass[
	a4paper,									% paper format
	11pt,										% fontsize
	twoside,									% double-sided
	openright,									% begin new chapter on right side
	notitlepage,									% use no standard title page
	parskip=half,								% set paragraph skip to half of a line
]{scrreprt}										% KOMA-script report
%---------------------------------------------------------------------------

\raggedbottom
\KOMAoptions{cleardoublepage=plain}						%Add header and footer on blank pages

%Load Standard Packages: 
%---------------------------------------------------------------------------
\usepackage[standard-baselineskips]{cmbright}				%Fonts for math equation
\usepackage[french]{babel}							%French hyphenation
	%\usepackage[latin1]{inputenc}  					% Unix/Linux - load extended character set (ISO 8859-1)
	%\usepackage[applemac]{inputenc}
	%\usepackage[ansinew]{inputenc}  					% Windows - load extended character set (ISO 8859-1)
%\usepackage[utf8]{inputenc}							%Translate input in latex language
\usepackage[T1]{fontenc}							%Allow a good hyphenation with accentuated language
%\usepackage{tgbonum}
\usepackage{ae}									%Allow vectorial letters
\usepackage{lmodern}
\usepackage{fancyhdr}								%Allow manipulation on headers and tops
\usepackage{graphicx}								%Integration of images
\usepackage{float}									%Better integration of floating objects(tables, etc)
\usepackage{caption}								%For captions of figures and tables
\usepackage{booktabs}								%Nicer tables
\usepackage{tocvsec2}								%Means of controlling the sectional numbering
\usepackage{verbatim}								%Integration of source code
\usepackage{moreverb}								%Extension of verbatim
\usepackage{listings}								%Integration of code in LATEX 
\usepackage{multirow}								%Tables with multiple rows
\usepackage{pdfpages}						
\usepackage{pst-all}								%Better handling of texts and images
\usepackage{mathrsfs}
\usepackage{colortbl}
\usepackage{listings}
\usepackage{minted}								%Better colorizing of source code WARNING need to be set up correctly according to the documentation http://tug.ctan.org/macros/latex/contrib/minted/minted.pdf
\usepackage{ragged2e}
\captionsetup{font=small}



\newcolumntype{R}[1]{>{\raggedleft\arraybackslash }b{#1}}
\newcolumntype{L}[1]{>{\raggedright\arraybackslash }b{#1}}
\newcolumntype{C}[1]{>{\centering\arraybackslash }b{#1}}		%Adding new column types

%---------------------------------------------------------------------------

%Load Math packages
%---------------------------------------------------------------------------
\usepackage{amsmath}                    				   	% various features to facilitate writing math formulas
\usepackage{amsthm}                       	 				% enhanced version of latex's newtheorem
\usepackage{amsfonts}                      					% set of miscellaneous TeX fonts that augment the standard CM
										
\usepackage{amssymb}							% mathematical special characters
\usepackage{exscale}							% mathematical size corresponds to textsize
\usepackage{listings}
\usepackage{tikz,pgfplots}
\usepackage{array}
%---------------------------------------------------------------------------

%QR Code
%---------------------------------------------------------------------------
\usepackage{qrcode}
\usepackage{subcaption}							%Add sub-caption easily
%---------------------------------------------------------------------------

% Package to facilitate placement of boxes at absolute positions
%---------------------------------------------------------------------------
%\usepackage[absolute]{textpos}
\usepackage[absolute,overlay]{textpos}
\setlength{\TPHorizModule}{1mm}
\setlength{\TPVertModule}{1mm}
%---------------------------------------------------------------------------	

% Definition of Colors
%---------------------------------------------------------------------------
\RequirePackage{color}							% Color (not xcolor!)
\definecolor{linkblue}{rgb}{0,0,0.8}            				% Standard
\definecolor{darkblue}{rgb}{0,0.08,0.45} 				% Dark blue
\definecolor{brickred}{cmyk}{0,0.89,0.94,0.28} 			% Brickred
\definecolor{linkcolor}{rgb}{0,0,0}        					% Black for the print-version!
\definecolor{PEjaune}{rgb}{1,0.84,0}        				% Jaune PE
\definecolor{PEvert}{rgb}{0.14,0.5,0}        				% Vert PE

\definecolor{VertVAUD}{rgb}{0.054, 0.662, 0.301} %14, 169, 77}

%---------------------------------------------------------------------------

% Hyperref Package (Create links in a pdf)
%---------------------------------------------------------------------------
\usepackage[
	pdftex,frenchb,bookmarks,plainpages=false,pdfpagelabels,
	backref = {false},							% No index backreference
	colorlinks = {true},							% Color links in a PDF
	hypertexnames = {true},						% no failures "same page(i)"
	bookmarksopen = {true},						% opens the bar on the left side
	bookmarksopenlevel = {0},					% depth of opened bookmarks
	pdftitle = {Rapport Projet P2213},		   		% PDF-property
	pdfauthor = {Surdez Quentin},        				% PDF-property
	pdfsubject = {Promotion 21-22},        				% PDF-property
	linkcolor = {linkcolor},              					% Color of Links
	citecolor = {linkcolor},              					% Color of Cite-Links
	urlcolor = {linkblue},               					% Color of URLs
]{hyperref}
%---------------------------------------------------------------------------

% Set up page dimension
%---------------------------------------------------------------------------
\usepackage[
	a4paper,
	left=30mm,
	right=30mm,
	top=30mm,
	headheight=20mm,
	headsep=10mm,
	textheight=242mm,
	footskip=15mm
]{geometry}
\setlength\parindent{20pt}
%---------------------------------------------------------------------------

% Makeindex Package
%---------------------------------------------------------------------------
\usepackage{makeidx}                         					% To produce index
\makeindex                                    					% Index-Initialisation
%---------------------------------------------------------------------------

% Intro:
\pgfplotsset{compat=1.18} 
%---------------------------------------------------------------------------
\begin{document}                              					% Start Document
\settocdepth{subsection}									% Set depth of toc
\pagenumbering{Roman}														
%---------------------------------------------------------------------------

%Set up header and footer
%---------------------------------------------------------------------------
\fancyhf{}												%clean all fields
\fancypagestyle{plain}{									%new definition of plain style
	\fancyfoot[OR, EL]{\footnotesize \thepage}			%footer right part --> page number
	\fancyfoot[OL, ER]{\footnotesize \leftmark}			%footer left part --> chapter
	\fancyfoot[CE, CO]{P2213, QS \& RD}
	\fancyhead[C]{
	\begin{textblock}{0}[0, 0](10, 8)						%header center part --> logo CPNV + MCT 
		\includegraphics[scale=.7]{img/logoCPNV.png}
	\end{textblock}
	\begin{textblock}{0}[0, 0](175, 3)
		\includegraphics[scale=.5]{img/logoMCT.jpg}
	\end{textblock}
	}
}

\renewcommand{\chaptermark}[1]{\markboth{\thechapter.  #1}{}}
\renewcommand{\headrulewidth}{0pt}				% no header stripline
\renewcommand{\footrulewidth}{0pt} 				% no bottom stripline
%\renewcommand\listoflistingscaption{Liste des codes sources}


\pagestyle{plain}
\let\cleardoublepage\clearpage
%---------------------------------------------------------------------------

%=============================================================================================
% Page principale
%=============================================================================================
%---------------------------------------------------------------------------
\begin{titlepage}
	\setlength{\unitlength}{1mm}
%	\begin{textblock}{230}(-10,-10)
%		\begin{picture}(230,35)%32)
%			\put(73,0){\color{VertVAUD}\rule{160mm}{40mm}}
%		\end{picture}
%	\end{textblock}

	\begin{textblock}{0}[0,0](5,12) % (x,y)
		\includegraphics[scale=1]{img/logoCPNV.png}
	\end{textblock}

	\begin{textblock}{0}(158, 4)
		\includegraphics[scale=.7]{img/logoMCT.jpg}
	\end{textblock}





% Titre / Sous-titre / Auteur / Image de garde:
%---------------------------------------------------------------------------
	
	\flushleft
	\vspace*{1cm}
	%\fontfamily{cmr}\selectfont			%To have the default font
	\fontsize{18pt}{20pt}\selectfont
	CPNV - Centre Professionnel du Nord Vaudois \\
	\fontsize{12pt}{15pt}\selectfont\vspace{0.5em}
	MCT - Modules complémentaires techniqeus

	\vspace{3cm}

	\fontsize{30pt}{32pt}\selectfont 
	\noindent \textbf{Rapport de projet} \\

	\fontsize{18pt}{20pt}\selectfont\vspace{0.3em} P2213 \\

	\vspace{4cm}
	\fontsize{12pt}{15pt}\selectfont
	\begin{tabbing}
		xxxxxxxxxxxxxxx\=xxxxxxxxxxxxxxxxxxxxxxx \kill
		Rédacteur:\> Quentin Surdez\\ \\
		Relecture:\> Rafael Dousse\\ \\
		École:\> CPNV\\ \\
		Date:\> Yverdon-Les-Bains, le \today \\
	\end{tabbing}
\end{titlepage}
%---------------------------------------------------------------------------

%===========================================
% Table des matières
%===========================================
\tableofcontents

\listoffigures									% Table des figures
\listoftables									% Table des tableaux
%\listoflistings % Now typeset the list
\cleardoublepage
%---------------------------------------------------------------------------

%=============================================================================================
% Introduction
%=============================================================================================
\pagenumbering{arabic}
\setcounter{page}{1}

\chapter{Introduction}

Notre projet P2213, Robot Autonome, est la construction d'un robot qui puisse remplir plusieurs tâches
sans avoir d'intéractions avec l'homme. La tâche principale est le suivi de ligne. Nous avons d'autres 
tâches comme la reconnaissance de code QR pour effectuer certains programmes ou le contrôle à distance. \par

Ce document détaillera les choix effectués dans les catégories suivantes : 

\begin{itemize}
	\item Administratif
	\item Mécanique
	\item Électronique
	\item Informatique
\end{itemize}



\chapter{Administratif}

Nous avons d'abord commencé à utiliser les outils de la suite Office365 pour faire notre administration. 
En cours de route, nous avons fait le choix de changer et de passer à \LaTeX pour faire nos documents. 
Vous trouverez ci-dessous un tableau comprenant nos critères pour ce choix. 

\begin{table}[h!]
    \begin{center}
        \vspace{5mm}
        \label{tab:table1}
        \begin{tabular}{c|c|c} % <-- Alignments: 1st column left, 2nd middle and 3rd right, with vertical lines in between
            \toprule
            \textbf{ } & \textbf{Office365} & \textbf{\LaTeX}\\
            \midrule
            Installation & 8 & 5\\
            Facilité de prise en main & 8 & 3\\
            Documentation & 8 & 8 \\
            Produit créé & 5 & 8 \\
			Création de nouvelles compétences & 3 & 9\\
			\midrule
			Total & 32 & 33\\
            \bottomrule
			
        \end{tabular}
    \end{center}    
	\caption{Comparaison des différents outils administratifs, échelle 1 à 10 (1 le moins bon, 10 le meilleur)}
\end{table}

Nous pouvons observer que les deux outils se valent pour nous. La vraie différence réside dans la création
de nouvelles compétences en apprenant \LaTeX. Nous continuons d'utiliser la suite Office365 pour faire 
des graphiques avec Visio ou Excel pour des tableaux avec un nombre conséquent de données. \par

Cependant, Word a été mis de côté au profit de \LaTeX\ pour la documentation. Cela nous permet d'avoir
un plus grand contrôle de notre document et une justification du texte. Qui plus est, Office365 
est payant alors que \LaTeX\ est un projet Open-Source. \par

\chapter{Mécanique}

Notre robot a une forte partie mécanique. Nous avons du créer toutes les pièces pour construire un ensemble
cohérent qui puisse rouler. Nous avons souhaité avoir une directive de base, créer des pièces nous permettant
d'avoir de la place. Une strucure large, pour y intégrer les différents composants que nous souhaiterions intégré. \par

\section{Extension arbre moteur}

La première pièce à avoir eu plusieurs itérations est l'extension de l'arbre moteur. Nous avions besoin d'un 
arbre moteur plus grand que ceux déjà présent sur les moteurs. L'arbre du moteur fait 15mm et nous souhaitons 
que les roues ne rentrent pas en collision avec les autres éléments constituants le système de rotation. \par

Ce système est composé du moteur, d'une roue compteuse qu'on peut apercevoir dans l'image qui suit, de l'extension
de l'arbre moteur et enfin d'une roue. Pour être certain que les roues ne rentrent en collision nous avons choisi
de faire un arbre moteur de 45mm de long. Cette longueur nous offre la possibilité d'avoir de larges roues tout en ayant 
un port-à-faux réduit au minimum. \par

\begin{figure}[!ht]
	\centering
	\includegraphics[scale=.1]{img/ExtensionArbreMoteur.jpg}
	\label{ExtensionRound}
	\caption{Extension d'arbre moteur ronde}
\end{figure}

La première tentative a été celle qu'on peut voir ci-dessus. L'extension était ronde avec un embout 
plus fin. Pour intégrer les roues cela n'était pas stable. Nous nous sommes alors dirigé vers une extension
hexagonale. En effet, les roues possèdent d'un côté une chambre hexagonale permettant l'insertion de l'extension. \par

\begin{figure}[!h]
	\centering 
	\includegraphics[scale=.1]{img/ExtensionArbreMoteurHexa.jpg}
	\label{ExtensionHexa}
	\caption{Extension d'arbre moteur hexagonale}	
\end{figure}

\newpage
L'extension hexagonale s'intègre correctement dans nos roues. Elle est facilement usinable avec la bonne pince. \par

\begin{table}[!ht]
    \begin{center}
        \vspace{5mm}
        \label{tab:table2}
        \begin{tabular}{c|c|r} % <-- Alignments: 1st column left, 2nd middle and 3rd right, with vertical lines in between
            \toprule
            \textbf{ } & \textbf{Extension Ronde} & \textbf{Extension Hexagonale}\\
            \midrule
            Temps d'usinage & 5 & 8\\
            Compatibilité & 5 & 10\\
            Maintien & 8 & 8 \\
			\midrule
			Total & 18 & 26\\
            \bottomrule
        \end{tabular}
    \end{center}    
	\caption{Comparaison des différentes extensions, échelle 1 à 10 (1 le moins bon, 10 le meilleur)}
\end{table}


Ce tableau nous permet de voir les avantages offerts par l'extension hexagonale. Cela est pourquoi nous avons 
choisi de continuer avec des extension hexagonales. 

\newpage
\section{Roues compteuses}

Nous avons faits plusieurs itérations pour les roues compteuses. Ces itérations sont intriquement liées au 
changement dans le code du PID. Nos roues compteuses ont toujours fait le même diamètre, 45mm, mais le nombre
de trous les composant a changé avec le temps. 

\begin{figure}[!h]
	\centering
	\includegraphics[scale=.1]{img/ExtensionArbreMoteur.jpg}
	\label{RoueCompteuse}
	\caption{Roue compteuse avec 16 trous}
\end{figure}

Nous voyons notre roue compteuse avec 16 trous. Nous avons une autre version avec 24 trous. La voici : 

\begin{figure}[!h]
	\centering
	\includegraphics[scale=.08]{img/RoueCompteuse.png}
	\label{RoueCompteuse2}
	\caption{Roue compteuse avec 24 trous}
\end{figure}

Nous avons choisi la variante avec 24 trous, car c'est celle qui fonctionne le mieux avec notre code du PID.
Elle permet d'avoir, même lorsque le calcul du PID se fait à haute fréquence, d'avoir au moins une incrémentation
entre deux calculs. \par

\section{Plaque de Fondation}

Notre plaque de support, appelée "Foundation", a subi plusieurs itérations. La première les moteurs étaient en dessous
et le pont H et les batteries en dessus. Maintenant, les moteurs, les batteries et le pont H sont au-dessus. 
Cela permet de protéger les composants d'un choc avec un obstacle. \par

\begin{figure}[!ht]
	\centering 
	\includegraphics[scale=.5]{img/FoundationV0.png}
	\label{FoundationV0}
	\caption{Première itération "Foundation"}	
\end{figure}

Ici, nous pouvons voir la première version de notre plaque fondation. Les moteurs étaient en dessous tout comme 
les capteurs pour les roues compteuses. Cela ne garantit par leur intégrité lors du fonctionnement. \par

\begin{figure}[!ht]
	\centering 
	\includegraphics[scale=.4]{img/FoundationV2.0.png}
	\label{FoundationV2}
	\caption{Deuxième itération "Foundation"}	
\end{figure}

Ci-dessus, nous avons la deuxième version de la plaque "Foundation". Cette-fois ci les composants sont en-dessus
et seules les roues ont une partie qui touche le sol en dessous de la plaque.

\begin{table}[!ht]
    \begin{center}
        \vspace{5mm}
        \label{tab:table4}
        \begin{tabular}{c|c|c} % <-- Alignments: 1st column left, 2nd middle and 3rd right, with vertical lines in between
            \toprule
            \textbf{ } & \textbf{Plaque avec moteurs dessous} & \textbf{Plaque avec moteurs en dessus}\\
            \midrule
            Protection & 5 & 8\\
            Maintien des composants & 5 & 10\\
            Répartition du poids & 3 & 8 \\
			\midrule
			Total & 13 & 26\\
            \bottomrule
        \end{tabular}
    \end{center}    
	\caption{Comparaison des différentes plaques de fondation, échelle 1 à 10 (1 le moins bon, 10 le meilleur)}
\end{table}


Nous pouvons voir que la deuxième version remplit mieux nos critères. Principalement par le fait que les éléments
sont protégés des obstacles externes. La répartition du poids est aussi meilleure, tous les éléments lourds sont
le plus bas possible. \par

\section{Support moteur}

Les moteurs DC que nous avons choisi, soit les MODELCRAFT, étaient assemblés avec une pièce en aluminium permettant de
les soutenir sur une plaque. Cette pièce était déjà prête et ne nécessitait pas d'être refaite. \par

Nous avons, au cours de notre projet, fait une pièce y ressemblant mais en PMMA. Cela nous a permis de revoir quelques
points qui ne nous satifaisaient pas sur la pièce en aluminium déjà faite. \par

\begin{table}[!ht]
    \begin{center}
        \vspace{5mm}
        \label{tab:table5}
        \begin{tabular}{c|c|c} % <-- Alignments: 1st column left, 2nd middle and 3rd right, with vertical lines in between
            \toprule
            \textbf{ } & \textbf{Pièce refaite en PMMA} & \textbf{Pièce en aluminium}\\
            \midrule
            Compatibilité & 10 & 5\\
            Solidité & 7 & 10\\
            Usinage & 5 & 3\\
			\midrule
			Total & 22 & 18\\
            \bottomrule
        \end{tabular}
    \end{center}    
	\caption{Comparaison des différents support de moteurs, échelle 1 à 10 (1 le moins bon, 10 le meilleur)}
\end{table}

Nous pouvons voir que le principal atout de la pièce en PMMA est sa compatibilité avec notre système. En effet, nous l'avons 
créé pour que sa structure soit celle dont nous avons besoin pour notre projet. Ainsi, ses trous pour la visser sur les plaques
sont alignés avec les trous des plaques et sa hauteur correspond à la hauteur que nous souhaitons pour l'espacement entre les 
plaques. Son usinage prend du temps, comme les trous doivent être alignés. \par

La pièce en aluminium est surtout plus robuste que la pièce en plastique. Son usinage prend plus de temps et de compétence 
comme une CNC doit être utilisée. Notre système n'a pas pour but d'être robuste aux chocs et nous ne voyons donc pas l'intérêt
d'utiliser cette pièce. \par

\newpage
\section{Support Caméra}

Le support pour la caméra possède plusieurs contraintes importantes. La caméra doit vibrer le moins possible
pour avoir une qualité d'image suffisante pour pouvoir la traiter. Nous avons besoin d'une pièce robuste qui 
pourra répondre aux différentes contraintes que le support possède. \par

\begin{figure}[!ht]
	\centering 
	\includegraphics[scale=.1]{img/SupportCameraPI.png}
	\label{SupportCameraPI}
	\caption{Support CameraPI}	
\end{figure}

Nous pouvons voir ici une première pièce. Cependant, elle est longue. Donc plus propice à bouger par les vibrations. 
Le module pour fixer la caméra est ajouté sur l'échelle et cela peut créer des vibrations supplémentaires. \par

Nous avons donc décidé de la faire plus courte et d'une seule pièce pour éliminer ces sources de transfert de vibrations. 
Pour ce faire nous avons envoyé à l'imprimante 3D. \par

\begin{figure}[!h]
	\centering 
	\includegraphics[scale=.4]{img/SupportCameraPIV2.png}
	\label{SupportCameraPIV2}
	\caption{Pilier CameraPI 3D}	
\end{figure}

Les contraintes ont changé le développement de la pièce pour qu'elle s'y adapte. Nous avons donc décidé de 
la fusionner en une pièce et de l'imprimer en 3D. \par

\chapter{Électronique}

Nous avons réalisé plusieurs PCBs différents pour notre projet. Ils ont tous été faits par le CPNV sauf 
un qui est le PCB du pont en H. Une explication succinte accompagnera chaque présentation de PCB et une 
explication de nos choix concernant les différents composants. \par

\section{Pont en H}

\begin{figure}[!ht]
	\centering
	\includegraphics[scale=.7]{img/PontH.png}
	\label{PontH}
	\captionbelow{3D du pont H vue de haut}
\end{figure}

Le PCB pont H a été fait en se basant sur le moyen projet de notre collègue. 
Nous avons eu accès à tous ces documents pour pouvoir créer une nouvelle version. Plusieurs éléments n'ont pas été 
changé : 

\begin{itemize}
	\item Circuit du pont H, L298N
	\item Le choix d'ajouter des capacités de $470\mu F$ pour plus de sécurité
	\item Les capacités selon la datasheet du L298N de $100nF$ entre le signal et le ground ainsi qu'entre la puissance et le ground
	\item Le circuit TC74 pour capter la température du radiateur pour la communiquer via I2C
	\item Les 8 diodes de protection
\end{itemize}

Nous avons ajouté plusieurs composants : 

\begin{itemize}
	\item Un connecteur 2x10 en pin mâles
	\item Borniers pour le signal et la puissance
	\item Des indications pour informer sur l'utilisation ou la position des éléments
\end{itemize}

\begin{table}[!ht]
    \begin{center}
        \vspace{5mm}
        \label{tab:table6}
        \begin{tabular}{c|c|c} % <-- Alignments: 1st column left, 2nd middle and 3rd right, with vertical lines in between
            \toprule
            \textbf{ } & \textbf{Ancienne Version} & \textbf{Nouvelle version}\\
            \midrule
            Facilité d'utilisation & 8 & 5\\
            Grandeur & 8 & 5\\
			\midrule
			Total & 16 & 10\\
            \bottomrule
        \end{tabular}
    \end{center}    
	\caption{Comparaison des versions PCBs pont H, échelle 1 à 10 (1 le moins bon, 10 le meilleur)}
\end{table}

Les 8 diodes permettent de protéger le PCB contre les surtensions provenant du moteur lorsque ce dernier ne reçoit 
plus de courant et devient une génératrice en créant un champ en faisant tourner l'aimant. Le positionnement des 
diodes est comme le schéma ci-dessous. 

\begin{figure}[!h]
	\begin{center}
		\includegraphics[scale=.7]{img/Diodes.png}
		\label{Diodes}
		\captionbelow{Schéma électrique de l'emplacement des diodes}
	\end{center}
\end{figure}

Le schéma complet du pont H peut être trouvé ici : \href{run:./Schema_pont_H}{Schéma pont H}


\section{PCBs Alimentation}

La question de l'alimentation est primordial dans notre projet. Nous avons un système sur batterie avec plusieurs 
éléments demandant des courants plus ou moins grands. Nous avons donc dû dimensionner plusieurs alimentations pour 
répondre aux différents besoins. Tous les microcontrôleurs ainsi que les capteurs sont sur une alimentation pouvant 
fournir jusqu'à $0.5A$. Le raspberry ainsi que l'écran sont sur une alimentation pouvant fournir jusqu'à $4A$. Les
deux donnent un courant continu à $5V$. \par

\newpage
\subsection{5V/0.5A}

\begin{figure}[!h]
	\centering
	\includegraphics[scale=.7]{img/Alimentation5v0.5A.png}
	\label{Alim1A}
	\captionbelow{3D du PCB de l'alimentation 5V/0.5A}
\end{figure}

Ce PCB possède deux convertisseurs DC/DC qui permettent de transformer la puissance reçue des batteries en 
$5V$ stable à $0.5A$. Nous avons deux convertisseurs afin d'être sûr d'avoir suffisamment de courant pour alimenter
les différents composants sur le réseau. Le PCB est composé des composants suivants : 

\begin{itemize}
	\item Arduino Nano Every
	\item 2 Recom-78E5.0-0.5
	\item 4 condensateurs polarisés $10\mu F$, 2 par convertisseur
	\item 2 diodes 1N4007 entre la puissance et le 5V en sortie.
\end{itemize}

Nous avons décidé d'ajouter des boutons pour en avoir si jamais nous souhations les utiliser. Les entrées des borniers
sont à la fois pour donner la puissance aux capteurs et pour recevoir le signal qu'ils émettent. Nous avons ajouté 
des connecteurs 2x10 à pins mâles afin de pouvoir facilement connecter des consommateurs si nous souhaitons en 
ajouter. \par

Nous avons aussi installer un connecteur 1x3 qui nous permet de choisir d'où provient la puissance pour alimenter l'
Arduino. En effet, pour faciliter l'Upload depuis l'ordinateur, nous n'avons pas besoin d'enlever l'Arduino et de le 
remettre, mais simplement changer la connexion des pins. \par

\newpage
Nous avons aussi installé une LED directement lié à un convertisseur qui donne l'information si le PCB est en 
fonction ou non. Pour décider de sa résistance voici le calcul utilisé : 

\begin{figure}[!h]

		\[U_R = 5-2 = 3V\]
		\[R = \cfrac{U_R}{I_max} = \cfrac{3}{15_mA} = 0.2k\Omega\] 
		\caption{Équation pour trouver la résistance minimum pour LED}
		\label{eq2}

\end{figure}

Nous trouvons donc une valeur de $200\Omega$. Cependant, nous avons choisi une résistance de $470\Omega$, pour que 
la LED ne soit pas trop lumineuse.

Le schéma de ce PCB se trouve ici : \href{run:./Schema_Alim_0.5.pdf}{Schéma PCB Alimentation 5V/0.5A}


\subsection{5V/4A}

\begin{figure}[!h]
	\centering
	\includegraphics[scale=.7]{img/Alimentation5v4A.png}
	\label{Alim4A}
	\captionbelow{3D du PCB de l'alimentation 5V/4A}
\end{figure}

Nous pouvons voir que ce dernier est très sembable au PCB précédent. Ce PCB possède un convertisseur DC/DC $5V$ stable
et $4A$. Voici les raisons pour lesquels nous avons choisi ce convertisseur : 

\begin{itemize}
	\item Donne le courant pour le Raspberry PI, l'écran et la caméra
	\item Donne le courant aux LED strips qui sont dirigées par l'Arduino
\end{itemize}

L'étage auquel il se trouve sur le robot est celui consommant le plus. Nous avons donc dû choisir ce convertisseur.
La structure est semblable au précédent PCB. Cependant les borniers sont utilisés pour transmettre du $5V$ et du $GND$.
Nous les utilisons pour alimenter le Raspberry PI et les LEDs. \par

Nous avons une nouvelle fonction dans ce PCB qui est la transmission d'information via I2C entre le Raspberry PI et
l'Arduino Nano Every. Nous avons utilisé un level shifter pour garantir que le Raspberry PI ne sera pas en surtension. \par

Le schéma du PCB peut être trouvé ici : \href{run:./Schema_Alim_4.pdf}{Schéma PCB Alimentation 5V/4A}

\section{Support Capteurs}

\begin{figure}[!ht]
	\centering
	\includegraphics[scale=.5]{img/Capteur.png}
	\label{SupportCapteur}
	\captionbelow{3D du PCB support capteur}
\end{figure}

Le dernier des PCBs que nous avons créé est celui qui permet de supporter notre capteur à ultrasons. Nous y avons
ajouté les Arduino Nano Every que nous utilisons dans nos programmes. Il possède des borniers pour l'alimentation
et pour que le signal du capteur puisse être ammené au bon endroit. \par

Le schéma pour ce PCB se trouve ici : \href{run:./Schema_SupportCapteur.pdf}{Schéma du Support Capteur}


%1. Exposition de nos PCBs et schéma électrique ??? Est-ce qu'on doit en faire un ???
%2. Argumentation de nos choix de PCBs
%3. Approche des problèmes et méthodologies utilisées
%4. Développement électronique
%5. Direction choisie selon Datasheet ou non
%6. D'où proviennent les éléments

\chapter{Informatique}

Nous avons plusieurs programmes ayant des rôles majeurs. Nous avons tout d'abord le programme du 
microcontrôleur qui calcule le PID à une fréquence de 250Hz. Ensuite, nous avons le programme qui gère
la WebApp depuis le Raspberry. Ce programme en appele plusieurs autres qui gèrent les fonctions en lien 
avec la CameraPI. \par

La CameraPI communique par bus CSI avec le Raspberyy PI et le Raspberry PI communique avec le microcontrôleur
par bus I2C. Nous allons commencer par expliquer les choix faits dans le programme du PID, ensuite nous discuterons
des programmes qui gèrent la CameraPI pour finir sur le code de la WebApp. \par


\section{PID}

Nous avons besoin d'asservir nos moteurs avec un asservissement par PID. Nous souhaitons que le robot avance
tout droit lorsqu'on le lui en donne l'ordre. L'équation du PID en fonction du temps et de l'erreur calculée 
est la suivante : 

\begin{figure}[h]
	\[u=k_pe + k_i \int e \cdot dt + k_d \cfrac{de}{dt}\] 
	\caption{Équation du PID}
	\label{eq1}
\end{figure}

Nous avons utilisé cette équation pour mettre en place un calcul du PID dans notre système. Le $dt$ a été 
set up via des interruptions attachées. Une clock externe permet d'envoyer un signal à une fréquence de 250Hz
pour calculer le PID. \par

Nous avons deux possibilités pour déterminer les différents facteurs. Nous pouvons soit utiliser la méthode 
de Ziegler-Nichols ou alors la méthode empirique. Voici un tableau pour les départager : 

\begin{table}[!ht]
    \begin{center}
        \vspace{5mm}
        \label{tab:table7}
        \begin{tabular}{c|c|c} % <-- Alignments: 1st column left, 2nd middle and 3rd right, with vertical lines in between
            \toprule
            \textbf{ } & \textbf{Ziegler-Nichols} & \textbf{Empirique}\\
            \midrule
            Justesse des facteurs & 8 & 5\\
            Facilité d'application & 3 & 8\\
            \midrule
			Total & 11 & 13\\
            \bottomrule
        \end{tabular}
    \end{center}    
	\caption{Comparaison des différentes méthodes d'acquisition des facteurs PID, échelle 1 à 10 (1 le moins bon, 10 le meilleur)}
\end{table}

Les méthodes se valent entre elles. Cependant, la méthode via Ziegler-Nichols n'a pas été tout à fait comprite pour l'appliquer 
dans notre cas avec des moteurs DC. C'est pour cela que nous avons choisi de continuer avec la méthode empirique. \par


Vous trouverez de plus amples informations concernant le code et notre utilisation des interruptions
à ce fichier : \href{run:./Code_Arduino.pdf}{Explication code Arduino}

La méthode que nous avons choisie est celle empirique. Pour un détail de toutes les données utilisées 
et les graphiques correspondant veuillez vous référer au document suivant : Rapport PID \par



\section{CameraPI}

Pour utiliser la CameraPI nous avons dû commencer à apprendre à utiliser OpenCV. 
Open Source Computer Vision Library, est un logiciel open source de computer vision et de machine learning. 
Pour pouvoir l'installer sur un Raspberry PI, se référer à ce document : \href{run:./Installation_OpenCV.pdf}{Guide d'installation à OpenCV}\par

La découverte de ce logiciel s'est faite en plusieurs étapes : 

\begin{itemize}
	\item Son téléchargement, qui est plutôt difficile, si nous le téléchargeons depuis la source
	\item La première prise en main grâce à des tutoriels
	\item Création de projet correspondant à notre cahier des charges avec l'aide de tutoriels
	\item La prise en main des fonctions utilisées dans les projets pour les intégrer à notre code
\end{itemize}

La plupart des tutoriels ont été trouvé sur le site suivant : \href{https://pyimagesearch.com/}{pyimagesearch.com} \par

Le choix d'utiliser OpenCV s'est fait pour plusieurs raisons, dont plusieurs sont détaillées ici : \href{run:./Choix_Camera.pdf}{Choix de la caméra}. 
Cependant, voici une liste exhaustive de ce qui nous a fait choisir ce logiciel :

\begin{itemize}
	\item Il permet de remplir notre cahier des charges, avec même quelques ajouts de fonctionnalités
	\item Il est difficile d'approche, mais permet de développer une compétence intéressante
	\item Il nous permettra de faire du machine learning dans notre avenir
\end{itemize}

On peut voir que les raisons sont surtout personnelles. Un projet futur personnel serait d'utiliser Keras et TensorFlow
pour pouvoir apprendre le machine learning. \par

\section{WebApp}

La création de la WebApp s'est faite grâce au micro-framwork Flask. Nous avons construit le WebApp grâce aux langages 
HTML/CSS. Comme pour les autres inconnues de ce projet nous avons d'abord commencé à regarder des vidéos pour ensuite
s'approprier le langage et le comprendre. \par

Les raisons qui nous ont amenées à choisir Flask sont les suivantes : 

\begin{itemize}
	\item Son installation est simple et rapide 
	\item Il permet un transfert d'informations facilité entre la WebApp et le script python
	\item Il est compatible avec nos autres librairies, notamment OpenCV
	\item Il est un micro-framework et donc très léger
\end{itemize}

Il y a une hiérarchie importante lors de la communication pour établir la WebApp. Le script python appelle Flask et certaines 
de ses fonctions pour créer les pages Web. Il est donné un template HTML et ce dernier va chercher dans un dossier static 
notre code CSS. Ensuite, les infos de la WebApp sont renvoyés au script qui recommence une boucle. 
Voici un schéma pour représenter cette communication : 

\begin{figure}[!h]
	\centering
	\includegraphics[scale=.8]{img/Schema_WebApp.png}
	\label{WebApp}
	\caption{Schéma de la WebApp}
\end{figure}

La WebApp permet ainsi une communication à distance via un autre appareil connecté au même réseau que le serveur, soit 
le Raspberry PI. Pour plus d'informations à propos de la communication externe : \href{run:./Comm_Externe}{Communication externe} \par

\section{Communication}

La communication est un point important de notre projet. En effet, c'est grâce à la communication I2C que notre Raspberry PI peut 
échanger des informations avec l'Arduino Nano Every. C'est aussi une communication via WIFI qui nous permet de contrôler le robot 
à distance. \par

Chaque communication possède un document qui lui est lié voici les liens : \href{run:./Comm_Externe}{Communication externe} \&
\href{run:./Comm_interne}{Communication interne} \par

Nous allons donné les points clefs pour chaque communication et les raisons du choix de ces modes de communication. \par

Le bus de communication I2C permet une communication rapide entre un maître, Raspberry PI, et un ou plusieurs esclaves, les Arduinos. 
Ce bus de communication est rapide, soit ~100kbits/s, cela permet d'avoir une transmission quasi instantanée de l'information.
Les librairies d'un côté comme de l'autre sont simple à utiliser et se mette rapidement en place. 

\begin{itemize}
	\item Flexibilité pour ajouter des esclaves potentiels
	\item Rapide
	\item Facile à installer
\end{itemize}

Le protocole de communication WIFI est un protocole commun à presque tous les appareils connectés. Il est facile de se connecter 
à un WIFI avec nos téléphones ou nos ordinateurs. Ce protocole est sélectif. Il est construit de tel sorte que par défaut un 
password est requis pour avoir une connexion à un réseau commun. Cela permet d'éviter les interférences d'autres appareils. 

\begin{itemize}
	\item Protocole sécurisé par défaut 
	\item Facile à mettre en place
	\item Facile à utiliser
\end{itemize}

\chapter{Améliorations}

Les améliorations de notre projet sont réparties avec la même structure que le rapport lui-même. 
Ce sont des observations que nous avons faites au cours du développement. Ce sont certaines 
choses que nous aurions aimé faire, mais soit le temps nous a contraint, ou nos décisions 
ne nous permettaient pas de les faire. Certaines améliorations sont aussi au-delà de nos 
compétences actuelles et nous demanderaient beaucoup trop de temps. \par

\section{Mécanique}

La mécanique est le point dans lequel nous avons investi le moins de temps dans le projet. 
En effet, nous souhaitions nous diriger principalement sur l'aspect électronique et programmation
de notre projet. Ainsi, il existe des améliorations au niveau de la mécanique de notre projet. \par

Premièrement, notre robot n'est pas fermé. Le fait que sa connectique et ses moteurs ne soit 
pas protégé par une boîte ou des parois fait en sorte que les risques de dégâts restent 
relativement grand. \par

Nous avons aussi des pièces soutentant les moteurs dont la hauteur ne correspond pas à nos besoins. 
En effet, après l'ajout de nouvelles roues, nous avons dû réhausser le deuxième étage. Cela fragilise 
la structure de notre robot. \par

Le maintien des batteries n'est pas bon. Une cage d'entretoises les maintiennent dans un certain espace
et les empêche de trop bouger. Cependant, elles sont libres dans cette cage d'entretoises, donc bougent. 
Ensuite, l'accès aux batteries est simplement mauvais. Il faut soit avoir une grande dextérité pour 
extraire les batteries, ou dévisser les plaques du haut pour y accéder. \par

Un vérin électrique pour faire bouger l'avant ou l'arrière du robot aurait aussi été grandement 
apprécié \par

\section{Électronique}

Notre électronique souffre d'un manque d'organisation dans les connectiques inter-PCBs.
Notre connectique reste mal organisé et brouillon dans la structure du robot. \par

Nous aurions souhaité avoir un pilone de câble déservant les alimentations aux 
différents étages du robot. Une autre amélioration électronique serait d'organiser les 
connections inter-PCBs avec des connectiques dédiés et des câbles plats. Cela permettrait 
de s'assurer que les différents chocs que le robot subit ne défasse pas la connexion 
par les câbles. Cela permettrait aussi à l'ensemble d'être organisé et propre. \par

Notre PCB gérant la communication en I2C possède une particularité dont nous ne connaissons pas 
la cause. Au niveau de l'adaptateur de niveau logique, le GND du Raspberry PI ne doit pas être branché 
pour que le tout fonctionne correctement. Le problème n'est pas connu, il faudra donc s'y attarder
si le PCB est réutilisé. \par


\section{Informatique}

Notre installation d'OpenCV est lourde et lente, soit ~4Go et 2-4 heures d'installation
avec une optimisation de la valeur des SWAPFILES. Une amélioration potentielle serait de trouver
une installation moins gourmande en temps et aussi plus légère. Pour ce faire, il faudrait savoir
exactement les modules dont nous avons besoin pour nos différents programmes. \par

Une deuxième améiorations concernant notre code serait une optimisation de la régulation. 
En effet, nos différents facteurs PID peuvent être mieux déterminés par une méthode autre 
que celle empirique. Il serait nécessaire de faire des tests plus poussé pour avoir 
des graphiques représentant l'évolution de la vitesse et de la valeur de contrôle. Les facteurs
pourraient être alternés en fonction. La fréquence à laquelle le calcul est fait devrait aussi 
faire l'objet d'une étude poussée pour déterminer la meilleure avec nos restrictions. \par

\chapter{Conclusion}

Les différentes décisions prises ont été dans le but de remplir notre cahier des charges. 
Nous avons cherché à trouver des solutions pour pouvoir accomplir les différents objectifs que 
nous nous étions fixés en début de projet. \par

Il y a certaines décisions qui auraient pu être mises en place dès le début pour que nous ayons une
direction à suivre qui soit plus organisé, comme un tube central autour duquel les alimentations 
peuvent donner du courant. \par


\end{document}