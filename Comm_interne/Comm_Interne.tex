%	Auteur: Surdez Quentin
%	Titre:	 Étudiant MCT
%	Date: Mai 2022
%	Sujet: Rapport de projet P2213	
%============================================================
\documentclass[
	a4paper,									% paper format
	11pt,										% fontsize
	twoside,									% double-sided
	openright,									% begin new chapter on right side
	notitlepage,									% use no standard title page
	parskip=half,								% set paragraph skip to half of a line
]{scrreprt}										% KOMA-script report
%---------------------------------------------------------------------------

\raggedbottom
\KOMAoptions{cleardoublepage=plain}						%Add header and footer on blank pages

%Load Standard Packages: 
%---------------------------------------------------------------------------
\usepackage[standard-baselineskips]{cmbright}				%Fonts for math equation
\usepackage[french]{babel}							%French hyphenation
	%\usepackage[latin1]{inputenc}  					% Unix/Linux - load extended character set (ISO 8859-1)
	%\usepackage[applemac]{inputenc}
	%\usepackage[ansinew]{inputenc}  					% Windows - load extended character set (ISO 8859-1)
%\usepackage[utf8]{inputenc}							%Translate input in latex language
\usepackage[T1]{fontenc}							%Allow a good hyphenation with accentuated language
%\usepackage{tgbonum}
\usepackage{ae}									%Allow vectorial letters
\usepackage{lmodern}
\usepackage{fancyhdr}								%Allow manipulation on headers and tops
\usepackage{graphicx}								%Integration of images
\usepackage{float}									%Better integration of floating objects(tables, etc)
\usepackage{caption}								%For captions of figures and tables
\usepackage{booktabs}								%Nicer tables
\usepackage{tocvsec2}								%Means of controlling the sectional numbering
\usepackage{verbatim}								%Integration of source code
\usepackage{moreverb}								%Extension of verbatim
\usepackage{listings}								%Integration of code in LATEX 
\usepackage{multirow}								%Tables with multiple rows
\usepackage{pdfpages}						
\usepackage{pst-all}								%Better handling of texts and images
\usepackage{mathrsfs}
\usepackage{colortbl}
\usepackage{minted}
\usepackage{ragged2e}
\captionsetup{font=small}


\newcolumntype{R}[1]{>{\raggedleft\arraybackslash }b{#1}}
\newcolumntype{L}[1]{>{\raggedright\arraybackslash }b{#1}}
\newcolumntype{C}[1]{>{\centering\arraybackslash }b{#1}}		%Adding new column types

%---------------------------------------------------------------------------

%Load Math packages
%---------------------------------------------------------------------------
\usepackage{amsmath}                    				   	% various features to facilitate writing math formulas
\usepackage{amsthm}                       	 				% enhanced version of latex's newtheorem
\usepackage{amsfonts}                      					% set of miscellaneous TeX fonts that augment the standard CM
										
\usepackage{amssymb}							% mathematical special characters
\usepackage{exscale}							% mathematical size corresponds to textsize
\usepackage{listings}
\usepackage{tikz,pgfplots}
\usepackage{array}
%---------------------------------------------------------------------------

%QR Code
%---------------------------------------------------------------------------
\usepackage{qrcode}
\usepackage{subcaption}							%Add sub-caption easily
%---------------------------------------------------------------------------

% Package to facilitate placement of boxes at absolute positions
%---------------------------------------------------------------------------
%\usepackage[absolute]{textpos}
\usepackage[absolute,overlay]{textpos}
\setlength{\TPHorizModule}{1mm}
\setlength{\TPVertModule}{1mm}
%---------------------------------------------------------------------------	

% Definition of Colors
%---------------------------------------------------------------------------
\RequirePackage{color}							% Color (not xcolor!)
\definecolor{linkblue}{rgb}{0,0,0.8}            				% Standard
\definecolor{darkblue}{rgb}{0,0.08,0.45} 				% Dark blue
\definecolor{brickred}{cmyk}{0,0.89,0.94,0.28} 			% Brickred
\definecolor{linkcolor}{rgb}{0,0,0}        					% Black for the print-version!
\definecolor{PEjaune}{rgb}{1,0.84,0}        				% Jaune PE
\definecolor{PEvert}{rgb}{0.14,0.5,0}        				% Vert PE

\definecolor{VertVAUD}{rgb}{0.054, 0.662, 0.301} %14, 169, 77}

%---------------------------------------------------------------------------

% Hyperref Package (Create links in a pdf)
%---------------------------------------------------------------------------
\usepackage[
	pdftex,frenchb,bookmarks,plainpages=false,pdfpagelabels,
	backref = {false},							% No index backreference
	colorlinks = {true},							% Color links in a PDF
	hypertexnames = {true},						% no failures "same page(i)"
	bookmarksopen = {true},						% opens the bar on the left side
	bookmarksopenlevel = {0},					% depth of opened bookmarks
	pdftitle = {Rapport Projet P2213},		   		% PDF-property
	pdfauthor = {Surdez Quentin},        				% PDF-property
	pdfsubject = {Promotion 21-22},        				% PDF-property
	linkcolor = {linkcolor},              					% Color of Links
	citecolor = {linkcolor},              					% Color of Cite-Links
	urlcolor = {linkblue},               					% Color of URLs
]{hyperref}
%---------------------------------------------------------------------------

% Set up page dimension
%---------------------------------------------------------------------------
\usepackage[
	a4paper,
	left=30mm,
	right=30mm,
	top=30mm,
	headheight=20mm,
	headsep=10mm,
	textheight=242mm,
	footskip=15mm
]{geometry}
\setlength\parindent{20pt}
%---------------------------------------------------------------------------

% Makeindex Package
%---------------------------------------------------------------------------
\usepackage{makeidx}                         					% To produce index
\makeindex                                    					% Index-Initialisation
%---------------------------------------------------------------------------

% Intro:
\pgfplotsset{compat=1.18} 
%---------------------------------------------------------------------------
\begin{document}                              					% Start Document
\settocdepth{subsection}									% Set depth of toc
\pagenumbering{roman}														
%---------------------------------------------------------------------------

%Set up header and footer
%---------------------------------------------------------------------------
\fancyhf{}												%clean all fields
\fancypagestyle{plain}{									%new definition of plain style
	\fancyfoot[OR, EL]{\footnotesize \thepage}			%footer right part --> page number
	\fancyfoot[OL, ER]{\footnotesize \leftmark}			%footer left part --> chapter
	\fancyfoot[CE, CO]{P2213, QS \& RD}
	\fancyhead[C]{
	\begin{textblock}{0}[0, 0](10, 8)						%header center part --> logo CPNV + MCT 
		\includegraphics[scale=0.7]{img/logoCPNV.png}
	\end{textblock}
	\begin{textblock}{0}[0, 0](175, 3)
		\includegraphics[scale=0.5]{img/logoMCT.jpg}
	\end{textblock}
	}
}

\renewcommand{\chaptermark}[1]{\markboth{\thechapter.  #1}{}}
\renewcommand{\headrulewidth}{0pt}				% no header stripline
\renewcommand{\footrulewidth}{0pt} 				% no bottom stripline

\pagestyle{plain}
\let\cleardoublepage\clearpage
%---------------------------------------------------------------------------

%=============================================================================================
% Page principale
%=============================================================================================
%---------------------------------------------------------------------------
\begin{titlepage}
	\setlength{\unitlength}{1mm}
%	\begin{textblock}{230}(-10,-10)
%		\begin{picture}(230,35)%32)
%			\put(73,0){\color{VertVAUD}\rule{160mm}{40mm}}
%		\end{picture}
%	\end{textblock}

	\begin{textblock}{0}[0,0](5,12) % (x,y)
		\includegraphics[scale=1]{img/logoCPNV.png}
	\end{textblock}

	\begin{textblock}{0}(158, 2)
		\includegraphics[scale=0.8]{img/logoMCT.jpg}
	\end{textblock}





% Titre / Sous-titre / Auteur / Image de garde:
%---------------------------------------------------------------------------
	
	\flushleft
	\vspace*{1cm}
	%\fontfamily{cmr}\selectfont			%To have the default font
	\fontsize{18pt}{20pt}\selectfont
	CPNV - Centre Professionnel du Nord Vaudois \\
	\fontsize{12pt}{15pt}\selectfont\vspace{0.5em}
	MCT - Modules complémentaires techniqeus

	\vspace{3cm}

	\fontsize{30pt}{32pt}\selectfont 
	\noindent \textbf{Communication interne} \\

	\fontsize{18pt}{20pt}\selectfont\vspace{0.3em} P2213 \\

	\vspace{4cm}
	\fontsize{12pt}{15pt}\selectfont
	\begin{tabbing}
		xxxxxxxxxxxxxxx\=xxxxxxxxxxxxxxxxxxxxxxx \kill
		Rédacteur:\> Quentin Surdez\\ \\
		Relecture:\> Rafael Dousse\\ \\
		École:\> CPNV\\ \\
		Date:\> Yverdon-Les-Bains, le \today \\
	\end{tabbing}
\end{titlepage}
%---------------------------------------------------------------------------

%===========================================
% Table des matières
%===========================================
\tableofcontents

\listoffigures									% Table des figures
%\listoftables									% Table des tableaux
\cleardoublepage
%---------------------------------------------------------------------------

%=============================================================================================
% Introduction
%=============================================================================================
\pagenumbering{arabic}
\setcounter{page}{1}

\chapter{Introduction}
Ce document a pour but d'expliquer et de montrer notre évolution dans la communication interne du robot. 
La communication interne est la communication entre le Raspberry PI et l'Arduino Nano. 
Nous allons en premier lieu discuter des différents protocoles que nous avons pu essayer. Ensuite, nous parlerons
de son intégration dans le projet pour chaque protocole. Nous verrons dans la conclusion notre choix final et nous argumenterons ce choix. 

%=============================================================================================
% Doc
%=============================================================================================
\chapter{Construction}

La construction du protocole s'est faite de manière organique. Nous avons regardé plusieurs tutos sur internet 
pour nous familiariser avec les communications entre le Raspberry PI et l'Arduino Nano Every. Ces éléments sont les 
seuls nécessitants une communication particulière au sein de notre projet. 

\section{UART}

Nous avons commencé par nous intéresser au protocole de communication UART ou sérielle. Nous avons suivi un tuto
de : \href{https://www.aranacorp.com/fr/communication-serie-entre-raspberry-pi-et-arduino/}{aranacorp.com}. Ce tutoriel
nous a permis de prendre en main la communication. \par

Le code permettant de set up le Raspberry PI pour cette communication est quelque peu complexe. Le voici : 

\begin{figure}[!ht]
	
	\usemintedstyle{rainbow_dash}
	\begin{minted}{python}

import serial,time
if __name__ == '__main__':
    
    with serial.Serial("/dev/ttyACM0", 9600, timeout=1) as arduino:
        time.sleep(0.1) #wait for serial to open
        if arduino.isOpen():
            #print("{} connected!".format(arduino.port))
            try:
                while True:
                    #cmd=input("Enter command : ")
                    arduino.write(cmd.encode())


	\end{minted}
	\caption{Set up du RPI pour comm UART}
	\label{UARTcomm}
	\end{figure}

Nous pouvons observer que le set up est complexe et apparaît difficile à intégrer dans les différents programmes
que nous allons développer par la suite. Nous devons tout d'abord spécifier le port sur lequel l'Arduino est connecté. 
Ensuite, il faut écrire la commande. Du côté de l'Arduino, la librairie Serial est utilisée. \par

Le plus grand désavantage de cette communication est le temps que prennent les informations pour être échangées. 
En effet, la vitesse de l'UART est définie par le $baudrate$ utilisé. Cependant, l'Arduino a un petit buffer et 9600 est la valeur par défaut. 
Donc la vitesse est de 9600bit/s. 
Qui plus est, la library pyserial est limitée à la lecture et l'envoi d'un byte. Des paramètrages peuvent être effectués pour en lire et en envoyer plus. 
Cependant cela impact la vitesse d'envoi et de réception. \par


\section{I2C}

En prenant connaissance des limites de la communication UART, nous nous sommes intéressés à la communication 
via le protocole I2C. Nous avons pu découvrir le code via un tutoriel : \href{https://www.thegeekpub.com/18263/raspberry-pi-to-arduino-i2c-communication/}
{thegeekpub.com}. Cela nous a permis de prendre en main le code à utiliser pour set up la communication du côté 
du Raspberry PI et de l'Arduino. \par 

Voici le code du set up : 

\begin{figure}[!ht]
	
	\usemintedstyle{rainbow_dash}
	\begin{minted}{python}

import smbus from SMBus
	
addr = 0x08
bus = SMBus(1)

while True:
	bus.write_byte(addr, 1)


	\end{minted}
	\caption{Set up de l'I2C du RPI}
	\label{I2Ccomm}
	\end{figure}

Nous pouvons observer que du côté du Raspberry le set up est rapide et simple. Du côté de l'Arduino la librairie Wire
est utilisée. Voici le set up : 

\begin{figure}[!ht]
	
	\usemintedstyle{rainbow_dash}
	\begin{minted}{c}
#include<Wire.h>

void receiveEvent(int howMany)
{while (Wire.available())
  {int c = Wire.read();
	switch (c)
    {case 1:
        Enavant();
      break;}}}
void setup()
{ Wire.begin(0x8);
  Wire.onReceive(receiveEvent);}

	\end{minted}
	\caption{Set up de l'I2C de l'Arduino}
	\label{I2CcommArduino}
	\end{figure}

Nous pouvons voir que le set up en lui-même consiste de l'appel de deux fonction Wire.begin et Wire.onReceive. L'adresse
doit être la même que celle donnée dans le Raspberry PI. Ensuite la fonction receiveEvent() permet de déterminer selon
l'ordre reçu d'appeler telle ou telle fonction dans notre code. Cette fonctionnalité est détaillée dans le document "Explication Code Arduino". \par

La communication par I2C possède plusieurs avantages. Elle est plus rapide que l'UART, soit 100kbits/s. Elle est plus facile à intégrer dans les
programmes que nous créons. Elle est flexible, c'est-à-dire que nous pouvons intégrer plusieurs Arduino slaves si 
nous en avons le besoin dans le futur. \par

\chapter{Conclusion}

Nous avons pu observer les différents protocoles testés dans le cadre de notre projet. Nous avons choisi le protocole 
de communication I2C. Les raisons pour ce choix sont les suivantes : 

\begin{itemize}
	\item Rapidité de la transmission de l'information
 	\item Flexibilité pour ajouter des slaves qui communiqueraient avec le Raspberry
  	\item Intégration facilitée dans les codes que nous créons et aucune incompatibilité avec le module de caméra ou OpenCV
\end{itemize}

Tous les codes incorporant une communication critique, possède le set up pour l'I2C du côté de l'Arduino et du côté du
Raspberry PI. 

\end{document}