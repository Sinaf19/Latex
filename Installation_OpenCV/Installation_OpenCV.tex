%	Auteur: Surdez Quentin
%	Titre:	 Étudiant MCT
%	Date: Mai 2022
%	Sujet: Rapport de projet P2213	
%============================================================
\documentclass[
	a4paper,									% paper format
	11pt,										% fontsize
	twoside,									% double-sided
	openright,									% begin new chapter on right side
	notitlepage,									% use no standard title page
	parskip=half,								% set paragraph skip to half of a line
]{scrreprt}										% KOMA-script report
%---------------------------------------------------------------------------

\raggedbottom
\KOMAoptions{cleardoublepage=plain}						%Add header and footer on blank pages

%Load Standard Packages: 
%---------------------------------------------------------------------------
\usepackage[standard-baselineskips]{cmbright}				%Fonts for math equation
\usepackage[french]{babel}							%French hyphenation
	%\usepackage[latin1]{inputenc}  					% Unix/Linux - load extended character set (ISO 8859-1)
	%\usepackage[applemac]{inputenc}
	%\usepackage[ansinew]{inputenc}  					% Windows - load extended character set (ISO 8859-1)
%\usepackage[utf8]{inputenc}							%Translate input in latex language
\usepackage[T1]{fontenc}							%Allow a good hyphenation with accentuated language
%\usepackage{tgbonum}
\usepackage{ae}									%Allow vectorial letters
\usepackage{lmodern}
\usepackage{fancyhdr}								%Allow manipulation on headers and tops
\usepackage{graphicx}								%Integration of images
\usepackage{float}									%Better integration of floating objects(tables, etc)
\usepackage{caption}								%For captions of figures and tables
\usepackage{booktabs}								%Nicer tables
\usepackage{tocvsec2}								%Means of controlling the sectional numbering
\usepackage{verbatim}								%Integration of source code
\usepackage{moreverb}								%Extension of verbatim
\usepackage{listings}								%Integration of code in LATEX 
\usepackage{multirow}								%Tables with multiple rows
\usepackage{pdfpages}						
\usepackage{pst-all}								%Better handling of texts and images
\usepackage{mathrsfs}
\usepackage{colortbl}
\usepackage{listings}
\usepackage{minted}
\usemintedstyle{rainbow_dash}
\usepackage{ragged2e}
\captionsetup{font=small}



\newcolumntype{R}[1]{>{\raggedleft\arraybackslash }b{#1}}
\newcolumntype{L}[1]{>{\raggedright\arraybackslash }b{#1}}
\newcolumntype{C}[1]{>{\centering\arraybackslash }b{#1}}		%Adding new column types

%---------------------------------------------------------------------------

%Load Math packages
%---------------------------------------------------------------------------
\usepackage{amsmath}                    				   	% various features to facilitate writing math formulas
\usepackage{amsthm}                       	 				% enhanced version of latex's newtheorem
\usepackage{amsfonts}                      					% set of miscellaneous TeX fonts that augment the standard CM
										
\usepackage{amssymb}							% mathematical special characters
\usepackage{exscale}							% mathematical size corresponds to textsize
\usepackage{listings}
\usepackage{tikz,pgfplots}
\usepackage{array}
%---------------------------------------------------------------------------

%QR Code
%---------------------------------------------------------------------------
\usepackage{qrcode}
\usepackage{subcaption}							%Add sub-caption easily
%---------------------------------------------------------------------------

% Package to facilitate placement of boxes at absolute positions
%---------------------------------------------------------------------------
%\usepackage[absolute]{textpos}
\usepackage[absolute,overlay]{textpos}
\setlength{\TPHorizModule}{1mm}
\setlength{\TPVertModule}{1mm}
%---------------------------------------------------------------------------	

% Definition of Colors
%---------------------------------------------------------------------------
\RequirePackage{color}							% Color (not xcolor!)
\definecolor{linkblue}{rgb}{0,0,0.8}            				% Standard
\definecolor{darkblue}{rgb}{0,0.08,0.45} 				% Dark blue
\definecolor{brickred}{cmyk}{0,0.89,0.94,0.28} 			% Brickred
\definecolor{linkcolor}{rgb}{0,0,0}        					% Black for the print-version!
\definecolor{PEjaune}{rgb}{1,0.84,0}        				% Jaune PE
\definecolor{PEvert}{rgb}{0.14,0.5,0}        				% Vert PE

\definecolor{VertVAUD}{rgb}{0.054, 0.662, 0.301} %14, 169, 77}

%---------------------------------------------------------------------------

% Hyperref Package (Create links in a pdf)
%---------------------------------------------------------------------------
\usepackage[
	pdftex,frenchb,bookmarks,plainpages=false,pdfpagelabels,
	backref = {false},							% No index backreference
	colorlinks = {true},							% Color links in a PDF
	hypertexnames = {true},						% no failures "same page(i)"
	bookmarksopen = {true},						% opens the bar on the left side
	bookmarksopenlevel = {0},					% depth of opened bookmarks
	pdftitle = {Rapport Projet P2213},		   		% PDF-property
	pdfauthor = {Surdez Quentin},        				% PDF-property
	pdfsubject = {Promotion 21-22},        				% PDF-property
	linkcolor = {linkcolor},              					% Color of Links
	citecolor = {linkcolor},              					% Color of Cite-Links
	urlcolor = {linkblue},               					% Color of URLs
]{hyperref}
%---------------------------------------------------------------------------

% Set up page dimension
%---------------------------------------------------------------------------
\usepackage[
	a4paper,
	left=30mm,
	right=30mm,
	top=30mm,
	headheight=20mm,
	headsep=10mm,
	textheight=242mm,
	footskip=15mm
]{geometry}
\setlength\parindent{20pt}
%---------------------------------------------------------------------------

% Makeindex Package
%---------------------------------------------------------------------------
\usepackage{makeidx}                         					% To produce index
\makeindex                                    					% Index-Initialisation
%---------------------------------------------------------------------------

% Intro:
\pgfplotsset{compat=1.18} 
%---------------------------------------------------------------------------
\begin{document}                              					% Start Document
\settocdepth{subsection}									% Set depth of toc
\pagenumbering{roman}														
%---------------------------------------------------------------------------

%Set up header and footer
%---------------------------------------------------------------------------
\fancyhf{}												%clean all fields
\fancypagestyle{plain}{									%new definition of plain style
	\fancyfoot[OR, EL]{\footnotesize \thepage}			%footer right part --> page number
	\fancyfoot[OL, ER]{\footnotesize \leftmark}			%footer left part --> chapter
	\fancyfoot[CE, CO]{P2213, QS \& RD}
	\fancyhead[C]{
	\begin{textblock}{0}[0, 0](10, 8)						%header center part --> logo CPNV + MCT 
		\includegraphics[scale=0.7]{img/logoCPNV.png}
	\end{textblock}
	\begin{textblock}{0}[0, 0](175, 3)
		\includegraphics[scale=0.5]{img/logoMCT.jpg}
	\end{textblock}
	}
}

\renewcommand{\chaptermark}[1]{\markboth{\thechapter.  #1}{}}
\renewcommand{\headrulewidth}{0pt}				% no header stripline
\renewcommand{\footrulewidth}{0pt} 				% no bottom stripline

\pagestyle{plain}
\let\cleardoublepage\clearpage
%---------------------------------------------------------------------------

%=============================================================================================
% Page principale
%=============================================================================================
%---------------------------------------------------------------------------
\begin{titlepage}
	\setlength{\unitlength}{1mm}
%	\begin{textblock}{230}(-10,-10)
%		\begin{picture}(230,35)%32)
%			\put(73,0){\color{VertVAUD}\rule{160mm}{40mm}}
%		\end{picture}
%	\end{textblock}

	\begin{textblock}{0}[0,0](5,12) % (x,y)
		\includegraphics[scale=1]{img/logoCPNV.png}
	\end{textblock}

	\begin{textblock}{0}(158, 2)
		\includegraphics[scale=0.8]{img/logoMCT.jpg}
	\end{textblock}





% Titre / Sous-titre / Auteur / Image de garde:
%---------------------------------------------------------------------------
	
	\flushleft
	\vspace*{1cm}
	%\fontfamily{cmr}\selectfont			%To have the default font
	\fontsize{18pt}{20pt}\selectfont
	CPNV - Centre Professionnel du Nord Vaudois \\
	\fontsize{12pt}{15pt}\selectfont\vspace{0.5em}
	MCT - Modules complémentaires techniqeus

	\vspace{3cm}

	\fontsize{30pt}{32pt}\selectfont 
	\noindent \textbf{Installation OpenCV} \\

	\fontsize{18pt}{20pt}\selectfont\vspace{0.3em} P2213 \\

	\vspace{4cm}
	\fontsize{12pt}{15pt}\selectfont
	\begin{tabbing}
		xxxxxxxxxxxxxxx\=xxxxxxxxxxxxxxxxxxxxxxx \kill
		Rédacteur:\> Quentin Surdez\\ \\
		Relecture:\> Rafael Dousse\\ \\
		École:\> CPNV\\ \\
		Date:\> Yverdon-Les-Bains, le \today \\
	\end{tabbing}
\end{titlepage}
%---------------------------------------------------------------------------

%===========================================
% Table des matières
%===========================================
\tableofcontents

%\listoffigures									% Table des figures
%\listoftables									% Table des tableaux
\cleardoublepage
%---------------------------------------------------------------------------

%=============================================================================================
% Introduction
%=============================================================================================
\pagenumbering{arabic}
\setcounter{page}{1}

\chapter{Introduction}

\begin{figure}[!ht]
	\centering
	\includegraphics[scale=.2]{img/OpenCV.png}
	\label{OpenCV}
\end{figure}

OpenCV est un logiciel open-source permettant un traitement de l'image. Nous l'utilisons pour reconnaître des codes QR,
reconnaître une ligne et encore streamer le flux d'images sur une page Web. Nous l'avons installé en entier, ce que 
ne permet pas de faire un $pip\ install$. Pour installer OpenCV en entier, nous avons besoin d'utiliser CMake. 
Les différentes étapes sont détaillées dans ce document.

\chapter{Installation}

La première chose à checker est la version de l'OS utilisée. Nous avons besoin de Bullseye 32bits. 
Ensuite, nous pouvons lancer l'installation. Je n'ai pas réussi à installer via pip. J'ai uniquement réussi 
l'installation depuis la source. C'est cette installation que je vais expliquer pas à pas dans ce document. \par

La première chose à faire est de s'assurer que le système est à jour via ces deux commandes : 

\begin{minted}[linenos]{bash}
	sudo apt update
	sudo apt upgrade
\end{minted}

Nous devons maintenant agrandir le filesystem. Pour ce faire, il vous faut taper : 

\begin{minted}[linenos, firstnumber=last]{bash}
	sudo raspi-config
\end{minted}

\noindent puis aller dans les options avancées et la première option est d'agrandir le filesystem. 
Il vous faudra reboot le Raspberry PI pour que ce changement soit effectif.

Nous devons ensuite construire l'environnement avec des librairies qui nous permettront d'utiliser
OpenCV à sa pleine capacité avec notre Raspberry PI. Chaque batch de librairie sera expliqué. \par

Nous allons premièrement installé des developer tools, dont CMake, qui vont nous aider à build OpenCV : 

\begin{minted}[linenos, firstnumber=last]{bash}
	sudo apt-get install build-essential cmake pkg-config
\end{minted}

Nous allons maintenant télécharger des packages I/O  permettant de gérer plusieurs formats d'images. Ces 
fichiers seront supportés : JPEG, PNG, TIFF, etc. 

\begin{minted}[linenos, firstnumber=last]{bash}
	sudo apt-get install libjpeg-dev libtiff5-dev libjasper-dev libpng-dev
\end{minted}

Nous avons maintenant les packages qui gèrent les images, nous en avons aussi besoin pour gérer les vidéos : 

\begin{minted}[linenos, firstnumber=last]{bash}
	sudo apt-get install libavcodec-dev libavformat-dev libswscale-dev libv4l-dev
	sudo apt-get install libxvidcore-dev libx264-dev
\end{minted}

La librairie OpenCV vient avec un module appelé highgui qui est un module permettant d'afficher les images et 
les vidéos sur l'espace de travail. Nous allons donc devoir installé toutes les libairies GTK : 

\begin{minted}[linenos, firstnumber=last]{bash}
	sudo apt install libfontconfig1-dev libcairo2-dev
	sudo apt install libgdk-pixbuf2.0-dev libpango1.0-dev
	sudo apt install libgtk2.0-dev libgtk-3-dev
\end{minted}

\newpage
Beaucoup d'opérations dans OpenCV utilisent des matrices nous pouvons optimisé ces opérations avec quelques librairies, 
ces librairies sont vraiment importantes pour des systèmes avec des contraintes de ressources comme notre Raspberry PI : 
\begin{minted}[linenos, firstnumber=last]{bash}
	sudo apt-get install libatlas-base-dev gfortran
\end{minted}

Nous devons ensuite installer des librairies HD5 et QT GUIs pour faciliter l'installation, 
si vous travaillez avec une version Legacy de PI OS il vous faudra les versions des librairies qui 
correspondent à votre OS : 

\begin{minted}[linenos, firstnumber=last]{bash}
	sudo apt install libhdf5-dev libhdf5-serial-dev libhdf5-103
	sudo apt install libqt5gui5 libqt5webkit5 libqt5test5 python3-pyqt5
	sudo apt install libgstreamer-plugins-base1.0-dev libgstreamer1.0-dev
\end{minted}

Nous nous assurons que python3 est bien installé sur notre système et téléchargeons numpy pour python3: 

\begin{minted}[linenos, firstnumber=last]{bash}
	sudo apt-get install python3-dev python3-numpy
\end{minted}

Et nous pouvons enfin télécharger OpenCV depuis git, la première chose à faire est donc de s'assurer 
que git est bien installé sur notre système et ensuite nous téléchargeons OpenCV : 

\begin{minted}[linenos, firstnumber=last]{bash}
	sudo apt-get install git
	git clone https://github.com/opencv/opencv.git
\end{minted}

Cela va créer un directory opencv dans le home directory. Maintenant, nous devons créer un directory build 
dans le directory opencv : 

\begin{minted}[linenos, firstnumber=last]{bash}
	cd ~/opencv
	mkdir build
	cd build
\end{minted}

Et nous pouvons enfin installer OpenCV, attention la commande suivante doit être exécutée depuis le build folder :

\begin{minted}[linenos, firstnumber=last]{bash}
	cmake ../
\end{minted}

Nous pouvons vérifier que l'installation s'est bien déroulée en checkant l'output pour Python3 : 

\begin{lstlisting}{bash}
	--   Python 3:
	--     Interpreter: /usr/bin/python3.9 (ver 3.9.x)
	--     Libraries: /usr/lib/x86_64-linux-gnu/libpython3.4m.so 
	--     numpy: /usr/lib/python3/dist-packages/numpy/core/include 
	--     packages path: lib/python3.4/dist-packages
\end{lstlisting}

Pour que les prochains processus se fassent rapidement en utilisant toutes les ressources 
du Raspberry PI. Nous changeons la valeure des SWAPFILES. Cette valeure devra être remise à zéro 
à la fin de l'installation.

\begin{minted}[linenos, firstnumber=last]{bash}
	sudo nano /etc/dphys-swapfile
\end{minted}

\newpage
Éditer les fichiers directement : 

\begin{minted}[linenos, firstnumber=last]{bash}
	# set size to absolute value, leaving empty (default) then uses computed value
	# you most likely don't want this, unless you have an special disk situation
	# CONF_SWAPSIZE=100
	CONF_SWAPSIZE=2048
\end{minted}

Nous devons restart le swap service : 

\begin{minted}[linenos, firstnumber=last]{bash}
	sudo /etc/init.d/dphys-swapfile stop
	sudo /etc/init.d/dphys-swapfile start
\end{minted}

Et maintenant nous buildons les files pour pouvoir utiliser OpenCV, nous précisons aussi le nombre de core à utiliser 
pour accélérer le processus : 

\begin{minted}[linenos, firstnumber=last]{bash}
	make -j4
	sudo make install
\end{minted}

La commande make peut prendre entre 2-4 heures. \par

Nous pouvons vérifier que l'installation s'est bien déroulée en tapant les commandes suivantes dans le terminal : 

\begin{minted}[linenos, firstnumber=last]{bash}
	python
	>>> import cv2
	>>> cv2.__version__
\end{minted}

Si l'installation s'est bien déroulée l'output de la commande vous donnera la version installée. (4.5.x). Cependant
il peut arriver qu'une erreur du genre NoModuleFound apparaisse. Dans ce cas, il faut éditer la variable PYTHONPATH : 

\begin{minted}[linenos, firstnumber=last]{bash}
	nano ~/.bashrc
	#Add to the end of the document the following path if OpenCV is there
	export PYTHONPATH=/usr/local/lib/python3.9/site-packages:$PYTHONPATH
	#Exit the nano text editor with Command + X, Y, enter
	source ~/.bashrc
\end{minted}

Revérifier l'installation par la commande précédente.

\chapter{Conclusion}

Le téléchargement de logiciel par la source est compliquée et requiert beaucoup de temps. Chaque étape doit être
suivie scrupuleusement suivie pour être certain d'avoir un output propre et exécutable. \par

La rédaction de ce document a été faite après avoir réussi à installer OpenCV sans le documenter. 
Réussir à le réinstaller nous a coûté beaucoup de temps, ~1 journée, ce qui aurait pu être 
investi dans l'avancée de notre projet. 

\end{document}